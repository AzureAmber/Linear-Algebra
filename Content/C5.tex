\newpage

\section[Day 5: Eigenvectors]{ Eigenvectors }

\subsection{ Diagonalization }

    \begin{definition}{Diagonalizable Matrices}{16cm}
        Matrix A $\in$ $M_{n \times n}(\mathbb{R})$
        is {\color{lblue} diagonal} if:

        \hspace{0.5cm}
        A =
        $\begin{bmatrix}
            a_{11} & 0 & \hdots & 0 \\
            0 & a_{22} & \hdots & 0 \\
            \vdots & \vdots & \ddots & \vdots \\
            0 & 0 & \hdots & a_{nn}
        \end{bmatrix}$
        \hspace{1cm}
        where $a_{11},...,a_{nn}$ $\in$ $\mathbb{R}$

        Then A is {\color{lblue} diagonalizable} if
        A is similar to a diagonal matrix $A_D$
    \end{definition}

    \vspace{0.5cm}



    \begin{wtheorem}{Diagonalizablility relationship between A
    and $A_{\mathcal{B}}$}{16cm}
        Let A $\in$ $M_{n \times n}(\mathbb{R})$.
        Then, A is diagonalizable if and only if
        for basis $\mathcal{B}$ = $v_1,...,v_n$ $\in$ $\mathbb{R}^n$, then
        $A_{\mathcal{B}}$ is diagonalizable.
    \end{wtheorem}

    \begin{proof}
        By {\color{red} theorem 2.5.4}, there is an invertible matrix B =
        $\begin{bmatrix}
            v_1 & ... & v_n
        \end{bmatrix}$ $\in$ $M_{n \times n}(\mathbb{R})$
        such that:
        
        \hspace{0.5cm}
        AB = B$A_{\mathcal{B}}$

        Thus, A is similar to $A_{\mathcal{B}}$

        \vspace{0.3cm}

        If A is diagonalizable, then A is similar to diagonal
        $A_D$ $\in$ $M_{n \times n}(\mathbb{R})$.

        Then by {\color{red} theorem 2.5.7},
        $A_{\mathcal{B}}$ is similar to $A_D$ and thus, $A_{\mathcal{B}}$
        is diagonalizable.

        \vspace{0.3cm}

        If $A_{\mathcal{B}}$ is diagonalizable, then $A_{\mathcal{B}}$
        is similar to diagonal $A_D$.
        
        Then by {\color{red} theorem 2.5.7},
        A is similar to $A_D$ and thus, A is diagonalizable.
    \end{proof}

    \vspace{0.5cm}



    \begin{wtheorem}{Diagonalizable A $\Leftrightarrow$ Av = $\lambda$v
    for some v}{16cm}
        A $\in$ $M_{n \times n}(\mathbb{R})$ is diagonalizable if and only if
        there is a basis $\mathcal{B}$ = $v_1,...,v_n$ $\in$ $\mathbb{R}^n$ where
        $A_{\mathcal{B}}$ is diagonal.
        Thus, for for some $\lambda_i$ $\in$ $\mathbb{R}$ where i = \{1,...,n\}:

        \hspace{0.5cm}
        $[Av_i]_{\mathcal{B}}$ = $[\lambda_i v_i]_{\mathcal{B}}$
    \end{wtheorem}

    \begin{proof}
        Suppose A is diagonalizable.
        Then there is an invertible B =
        $\begin{bmatrix}
            v_1 & ... & v_n
        \end{bmatrix}$, diagonal $A_D$
        $\in$ $M_{n \times n}(\mathbb{R})$
        such that AB = B$A_D$.
        By {\color{red} theorem 2.4.5}, the $\mathcal{B}$ = $v_1,...,v_n$ form a
        basis for $\mathbb{R}^n$. 
        By {\color{red} theorem 2.5.4}, then $A_D$ = $A_{\mathcal{B}}$
        so $A_{\mathcal{B}}$ is a diagonal.

        \vspace{0.3cm}

        Suppose there is a basis $\mathcal{B}$ = $v_1,...,v_n$ $\in$ $\mathbb{R}^n$
        where $A_{\mathcal{B}}$ is diagonal.

        By {\color{red} theorem 2.5.4}, then AB = B$A_{\mathcal{B}}$
        where B is invertible so A is diagonalizable.

        \vspace{0.3cm}

        By {\color{red} theorem 2.5.3}, then
        $A_{\mathcal{B}}$ =
        $\begin{bmatrix}
            [A(v_1)]_{\mathcal{B}} & ... & [A(v_n)]_{\mathcal{B}}
        \end{bmatrix}$.
        Since $A_{\mathcal{B}}$ is a diagonal, then there are
        $\lambda_i$ $\in$ $\mathbb{R}$ for i = \{1,...,n\} such that:

        \hspace{0.5cm}
        $[A(v_i)]_{\mathcal{B}}$ =
        $\begin{bmatrix}
            0 \\
            \vdots \\
            \lambda_i \\
            \vdots \\
            0
        \end{bmatrix}$
        = $\lambda_i e_i$
        = $\lambda_i [v_i]_{\mathcal{B}}$
        = $[\lambda_i v_i]_{\mathcal{B}}$
    \end{proof}

    \newpage



    \begin{definition}{Eigenvalues and Eigenvectors}{16cm}
        Let T: $\mathbb{R}^n$ $\rightarrow$ $\mathbb{R}^n$
        be a linear transformation.
        Then, nonzero v $\in$ $\mathbb{R}^n$ is an
        {\color{lblue} eigenvector} of T if T(v) = $\lambda$v
        for some {\color{lblue} eigenvalue} $\lambda$ $\in$ $\mathbb{R}$ of T.

        If eigenvectors $v_1,...,v_n$ $\in$ $\mathbb{R}^n$
        with eigenvalues $\lambda_1,...,\lambda_n$ form a basis, then
        $v_1,...,v_n$ is a {\color{lblue} eigenbasis} for T.
    \end{definition}

    \vspace{0.5cm}



    \begin{wtheorem}{Diagonalizable $\Leftrightarrow$ Existence of Eigenbasis}{16cm}
        A $\in$ $M_{n \times n}(\mathbb{R})$ is diagonalizable if and only if
        there is an eigenbasis $\mathcal{B}$ = $v_1,...,v_n$ with eigenvalues
        $\lambda_1,....,\lambda_n$ for A.
        Then for B =
        $\begin{bmatrix}
            v_1 & ... & v_n
        \end{bmatrix}$,
        $A_{\mathcal{B}}$ =
        $\begin{bmatrix}
            \lambda_1e_1 & ... & \lambda_ne_n
        \end{bmatrix}$
        $\in$ $M_{n \times n}(\mathbb{R})$:

        \hspace{0.5cm}
        AB = B$A_{\mathcal{B}}$
    \end{wtheorem}

    \begin{proof}
        Suppose A is diagonalizable. Then, by {\color{red} theorem 5.1.3},
        there is a basis $\mathcal{B}$ = $v_1,...,v_n$
        such that $A_{\mathcal{B}}$ is diagonal where
        $[Av_i]_{\mathcal{B}}$ = $[\lambda_i v_i]_{\mathcal{B}}$
        for some $\lambda_i$.
        Since $Av_i$ = $\lambda_i v_i$,
        then $v_i$ is an eigenvector with eigenvalue $v_i$.
        Thus, $\mathcal{B}$ = $v_1,...,v_n$ is an eigenbasis
        with eigenvalues $\lambda_1,....,\lambda_n$ for A.

        \vspace{0.3cm}

        Suppose there is an eigenbasis $\mathcal{B}$ = $v_1,...,v_n$ with
        eigenvalues $\lambda_1,....,\lambda_n$ for A.

        Since A$v_i$ = $\lambda_i v_i$,
        then $A_{\mathcal{B}}$ =
        $\begin{bmatrix}
            [Av_1]_{\mathcal{}} & ... & [Av_n]_{\mathcal{}}
        \end{bmatrix}$ =
        $\begin{bmatrix}
            [\lambda_1v_1]_{\mathcal{}} & ... & [\lambda_nv_n]_{\mathcal{}}
        \end{bmatrix}$ =
        $\begin{bmatrix}
            \lambda_1e_1 & ... & \lambda_ne_n
        \end{bmatrix}$,
        is diagonal.
        Thus, by {\color{red} theorem 5.1.3}, then A is diagonalizable.

        \vspace{0.3cm}

        Since $v_1,...,v_n$ is a basis, by {\color{red} theorem 2.5.4}, then
        AB = B$A_{\mathcal{B}}$.
    \end{proof}

    \vspace{0.5cm}



    \begin{wtheorem}{Eigenvalues of an Orthogonal matrix}{16cm}
        For orthogonal A $\in$ $M_{n \times n}(\mathbb{R})$,
        the only possible eigenvalues are $\pm 1$
    \end{wtheorem}

    \begin{proof}
        Since A is orthogonal, then $|Ax|$ = $|x|$.
        If $Ax$ = $\lambda x$, then:
        
        \hspace{0.5cm}
        $|x|$ = $|Ax|$ = $|\lambda x|$ = $|\lambda| |x|$

        Thus, $|\lambda|$ = 1 so $\lambda$ = $\pm 1$.
    \end{proof}

    \vspace{0.5cm}



    \begin{wtheorem}{Eigenvalues and Invertibility}{16cm}
        A $\in$ $M_{n \times n}(\mathbb{R})$ is invertible if and only if
        0 is not an eigenvalue of A
    \end{wtheorem}

    \begin{proof}
        Suppose A is invertible.
        Then by {\color{red} theorem 2.4.5}, the only solution to Ax = 0 is x = 0.
        Thus, if $\lambda$ = 0, then Ax = $\lambda$x = 0 has only x = 0 $\not =$ 0
        so $\lambda$ = 0 is not an eigenvalue.

        \vspace{0.3cm}

        Suppose $\lambda$ = 0 is not an eigenvalue of A.
        Then there are no nonzero x $\in$ $\mathbb{R}^n$
        such that Ax = $\lambda$x = 0 so only x = 0.
        Thus, by {\color{red} theorem 2.4.5}, A is invertible.
    \end{proof}

    \newpage





\subsection{ Characteristic Polynomial }

    \begin{wtheorem}{Determining Eigenvalues}{16cm}
        $\lambda$ $\in$ $\mathbb{R}$ is an eigenvalue of
        A $\in$ $M_{n \times n}(\mathbb{R})$ if and only if
        det($A - \lambda I_{n \times n}$) = 0
    \end{wtheorem}

    \begin{proof}
        Suppose $\lambda$ $\in$ $\mathbb{R}$ is an eigenvalue of A.
        Then there is a nonzero v $\in$ $\mathbb{R}^n$ such that Av = $\lambda$v
        so 0 = $Av - \lambda v$ = $(A - \lambda I_{n \times n})v$.
        Since nonzero v $\in$ ker($A - \lambda I_{n \times n}$),
        then $A - \lambda I_{n \times n}$ is not invertible.
        By {\color{red} theorem 4.2.3}, then det($A - \lambda I_{n \times n}$) = 0.

        \vspace{0.3cm}

        Suppose det($A - \lambda I_{n \times n}$) = 0.
        By {\color{red} theorem 4.2.3}, then $A - \lambda I_{n \times n}$
        is not invertible so there is a nonzero v $\in$ $\mathbb{R}^n$
        such that $(A - \lambda I_{n \times n})v$ = 0.
        Thus, Av = $\lambda$v so $\lambda$ is an eigenvalue.
    \end{proof}

    \vspace{0.5cm}



    \begin{wtheorem}{det($A - \lambda I_{n \times n}$)
    = det($A_{\mathcal{B}} - \lambda I_{n \times n}$)}{16cm}
        Let $\mathcal{B}$ = $v_1,...,v_n$ be a basis for $\mathbb{R}^n$. Then:

        \hspace{0.5cm}
        det(A) = det($A_{\mathcal{B}}$)
        \hspace{1cm}
        det($A - \lambda I_{n \times n}$)
        = det($A_{\mathcal{B}} - \lambda I_{n \times n}$)
    \end{wtheorem}

    \begin{proof}
        By {\color{red} theorem 2.5.4}, AB = B$A_{\mathcal{B}}$
        where B =
        $\begin{bmatrix}
            v_1 & ... & v_n
        \end{bmatrix}$
        is invertible. Thus:

        \hspace{0.5cm}
        det(A) = det($BA_{\mathcal{B}}B^{-1}$)
        = det(B)det($A_{\mathcal{B}}$)det($B^{-1}$)
        = det(B)det($A_{\mathcal{B}}$)(det(B))$^{-1}$
        = det($A_{\mathcal{B}}$)

        \vspace{0.3cm}

        For i = \{1,..,n\}:

        \hspace{0.5cm}
        $[(A - \lambda I_{n \times n})]_{\mathcal{B}}[v_i]_{\mathcal{B}}$
        = $[(A - \lambda I_{n \times n})v_i]_{\mathcal{B}}$
        = $[Av_i]_{\mathcal{B}} - [\lambda v_i]_{\mathcal{B}}$
        = $A_{\mathcal{B}}[v_i]_{\mathcal{B}} - \lambda[v_i]_{\mathcal{B}}$
        = $(A_{\mathcal{B}} - \lambda I_{n \times n})[v_i]_{\mathcal{B}}$

        Thus,
        det($A - \lambda I_{n \times n}$)
        = det($[A - \lambda I_{n \times n}]_{\mathcal{B}}$)
        = det($A_{\mathcal{B}} - \lambda I_{n \times n}$).
    \end{proof}

    \vspace{0.5cm}



    \begin{definition}{Characteristic Polynomial}{16cm}
        Since the entries, $a_{ij}^*$, of $A - \lambda I_{n \times n}$
        are the same as then entries of A except on its diagonal entries,
        $a_{ii}^*$ = $a_{ii} - \lambda$, then
        det($A - \lambda I_{n \times n}$)
        = $\sum_{\{i_1,...,i_n\}=\{1,...,n\}}$ $a_{1i_1},...,a_{ni_n}(-1)^I$
        contains an arrangement $a_{1i_1},...,a_{ni_n}(-1)^I$
        = $(a_{11}-\lambda)(a_{22}-\lambda)...(a_{nn}-\lambda)$
        which is a polynomial of degree n and the other arrangements
        contain less than n $(a_{ii} - \lambda)$ so
        det($A - \lambda I_{n \times n}$) is a polynomial of degree n
        with variable $\lambda$.

        \vspace{0.3cm}

        For A $\in$ $M_{n \times n}(\mathbb{R})$ and $\lambda$ $\in$ $\mathbb{R}$,
        then det($A - \lambda I_{n \times n}$) is the
        {\color{lblue} characteristic polynomial} of A.
    \end{definition}

    \vspace{0.5cm}



    \begin{definition}{Algebraic Multiplicity}{16cm}
        Let A $\in$ $M_{n \times n}(\mathbb{R})$ and $\lambda$ $\in$ $\mathbb{R}$.

        Then, eigenvalue $\lambda^*$ $\in$ $\mathbb{R}$ of A
        has an {\color{lblue} algebraic multiplicity} of k if:
        
        \hspace{0.5cm}
        det($A - \lambda I_{n \times n}$)
        = $(\lambda - \lambda^*)^k f(\lambda)$

        where $f(\lambda)$ is a polynomial of degree n-k with
        $f(\lambda^*)$ $\not =$ 0.
        Then, almu($\lambda^*$) = k $\geq$ 1.
    \end{definition}

    \vspace{0.5cm}



    \begin{wtheorem}{Relationship between Algebraic Multiplicity
    and Dimension}{16cm}
        For A $\in$ $M_{n \times n}(\mathbb{R})$, let
        $\lambda_1,...,\lambda_k$ $\in$ $\mathbb{R}$
        be distinct eigenvalues of A. Then k $\leq$ n where:

        \hspace{0.5cm}
        almu($\lambda_1$) + ... + almu($\lambda_k$) $\leq$ n

        If $\lambda_1,...,\lambda_k$ $\in$ $\mathbb{C}$,
        then k = n and almu($\lambda_1$) + ... + almu($\lambda_k$) = n.
    \end{wtheorem}

    \begin{proof}
        Since det($A - \lambda I_{n \times n}$) is a polynomial
        of degree n, then there are at most n real roots so k $\leq$ n.
        If $\lambda_1,...,\lambda_k$ $\in$ $\mathbb{R}$
        are distinct eigenvalues of A, then:

        \hspace{0.5cm}
        det($A - \lambda I_{n \times n}$)
        = $(\lambda - \lambda_1)^{m_1} ... (\lambda - \lambda_k)^{m_k} f(\lambda)$

        \hspace{0.5cm}
        almu($\lambda_1$) + ... + almu($\lambda_k$)
        = $m_1 + ... + m_k$
        $\leq$ n

        For $\lambda_1,...,\lambda_k$ $\in$ $\mathbb{C}$, then:

        \hspace{0.5cm}
        det($A - \lambda I_{n \times n}$)
        = $(\lambda - \lambda_1)^{m_1} ... (\lambda - \lambda_k)^{m_k}$

        \hspace{0.5cm}
        almu($\lambda_1$) + ... + almu($\lambda_k$)
        = $m_1 + ... + m_k$
        = n
    \end{proof}

    \newpage



    \begin{corollary}{Relationship between Algebraic Multiplicity
    and Determinant}{16cm}
        Let $\lambda_1,...,\lambda_k$ $\in$ $\mathbb{C}$ be distinct eigenvalues
        of A $\in$ $M_{n \times n}(\mathbb{R})$ with
        almu($\lambda_i$) = $m_i$ $\in$ $\mathbb{R}$.
        Then:

        \hspace{0.5cm}
        det(A) = $\lambda_1^{m_1}...\lambda_k^{m_k}$
    \end{corollary}

    \begin{proof}
        By {\color{red} theorem 5.2.5},
        almu($\lambda_1$) + ... + almu($\lambda_k$) = n. Thus:

        \hspace{0.5cm}
        det($A - \lambda I_{n \times n}$)
        = $(\lambda - \lambda_1)^{m_1} ... (\lambda - \lambda_k)^{m_k}$

        So for $\lambda$ = 0, then
        det(A) = $\lambda_1^{m_1}...\lambda_k^{m_k}$.
    \end{proof}

    \vspace{0.5cm}





\subsection{ Eigenspaces }

    \begin{definition}{Eigenspaces}{16cm}
        For A $\in$ $M_{n \times n}(\mathbb{R})$, let
        $\lambda$ $\in$ $\mathbb{R}$ be an eigenvalue.
        Then there is an eigenvector v $\not =$ 0 $\in$ $\mathbb{R}^n$
        such that Av = $\lambda$v so
        0 = $Av - \lambda v$ = $(A - \lambda I_{n \times n})v$.
        Thus, v $\in$ ker($A - \lambda I_{n \times n}$).

        Then, the {\color{lblue} eigenspace} of A for $\lambda$,
        the set of all eigenvectors v with eigenvalue $\lambda$:

        \hspace{0.5cm}
        $E_{\lambda}$ = ker($A - \lambda I_{n \times n}$)
    \end{definition}

    \vspace{0.5cm}



    \begin{definition}{Geometric Multiplicity}{16cm}
        Let A $\in$ $M_{n \times n}(\mathbb{R})$.
        Then, the {\color{lblue} geometric multiplicity}
        of eigenvalue $\lambda^*$ $\in$ $\mathbb{R}$ of A:
        
        \hspace{0.5cm}
        gemu($\lambda^*$)
        = dim(ker($E_{\lambda^*}$))
        = dim(ker($A - \lambda^* I_{n \times n}$))
        = nullity($A - \lambda^* I_{n \times n}$)

        Since $\lambda^*$ is an eigenvalue, then there is a eigenvector
        v $\not =$ 0 $\in$ ker($A - \lambda^* I_{n \times n}$)
        so dim(ker($A - \lambda^* I_{n \times n}$)) = gemu($\lambda^*$) $\geq$ 1.
    \end{definition}

    \vspace{0.5cm}



    \begin{wtheorem}{Relationship between Algebraic Multiplicity and
    Geometric Multiplicity}{16cm}
        Let $\lambda^*$ $\in$ $\mathbb{R}$ be an eigenvalue
        of A $\in$ $M_{n \times n}(\mathbb{R})$. Then:

        \hspace{0.5cm}
        gemu($\lambda^*$) $\leq$ almu($\lambda^*$)

        Thus, if $\lambda_1,...,\lambda_k$ $\in$ $\mathbb{R}$
        are distinct eigenvalues of A, then:

        \hspace{0.5cm}
        gemu($\lambda_1$) + ... + gemu($\lambda_k$) $\leq$ n
    \end{wtheorem}

    \begin{proof}
        Let gemu($\lambda^*$) = m so dim(ker($A - \lambda^* I_{n \times n}$)) = m.
        Thus, let $v_1,...,v_m$ form a basis for $E_{\lambda^*}$.

        Then choose $u_1,...,u_{n-m}$ $\in$ $\mathbb{R}^n$ such that
        $\mathcal{B}$ = $v_1,...,v_m,u_1,...,u_{n-m}$
        form a basis for $\mathbb{R}^n$.

        Since A$v_i$ = $\lambda^*v_i$ for i = \{1,...,m\}, then
        for $A_{\mathcal{B}}$ $\in$ $M_{n \times n}(\mathbb{R})$:

        \hspace{0.5cm}
        $A_{\mathcal{B}}$ =
        \footnotesize
        $\begin{bmatrix}
            \lambda^* & 0 & \hdots & 0 & a_{11} & a_{12} & ... & a_{1(n-m)} \\
            0 & \lambda^* & \hdots & 0 & a_{21} & a_{22} & ... & a_{2(n-m)} \\
            \vdots & \vdots & \ddots & \vdots & \vdots & \vdots & \ddots & \vdots \\
            0 & 0 & \hdots & \lambda^* & a_{m1} & a_{m2} & ... & a_{m(n-m)} \\
            0 & 0 & \hdots & 0 & a_{(m+1)1} & a_{(m+1)2} & ... & a_{(m+1)(n-m)} \\
            \vdots & \vdots & \ddots & \vdots & \vdots & \vdots & \ddots & \vdots \\
            0 & 0 & \hdots & 0 & a_{n1} & a_{n2} & ... & a_{n(n-m)} 
        \end{bmatrix}$
        \normalsize
        =
        \footnotesize
        $\begin{bmatrix}
            \lambda^* I_{m \times n} & B_{m \times (n-m)} \\
            0_{(n-m) \times (n-m)} & D_{(n-m) \times (n-m)}
        \end{bmatrix}$
        \normalsize

        where $[Au_j]_{\mathcal{B}}$ = $(a_{1j},...,a_{nj})$
        for j $\in$ \{1,...,n-m\}. Thus:

        \hspace{0.5cm}
        det($A_{\mathcal{B}} - \lambda I_{n \times n}$)
        = $(\lambda^* - \lambda)^m$det($D - \lambda I_{(n-m) \times (n-m)}$)
        
        By {\color{red} theorem 5.2.2}, then det($A - \lambda I_{n \times n}$)
        = det($A_{\mathcal{B}} - \lambda I_{n \times n}$).

        Thus, det($A - \lambda I_{n \times n}$)
        = $(\lambda^* - \lambda)^m$det($D - \lambda I_{(n-m) \times (n-m)}$)
        so gemu($\lambda^*$) = m $\leq$ almu($\lambda^*$).

        Then for eigenvalues $\lambda_1,...,\lambda_k$ $\in$ $\mathbb{R}$
        for A, by {\color{red} theorem 5.2.5}:

        \hspace{0.5cm}
        gemu($\lambda_1$) + ... + gemu($\lambda_k$)
        $\leq$ almu($\lambda_1$) + ... + almu($\lambda_k$)
        $\leq$ n
    \end{proof}
    
    \newpage



    \begin{wtheorem}{Vandermonde Matrix}{16cm}
        For $a_1,...,a_n$ $\in$ $\mathbb{R}$, the
        {\color{lblue} Vandermonde Matrix}, V $\in$ $M_{n \times n}(\mathbb{R})$:

        \hspace{0.5cm}
        V = 
        \footnotesize
        $\begin{bmatrix}
            1 & a_1 & a_1^2 & ... & a_1^{n-1} \\
            1 & a_2 & a_2^2 & ... & a_2^{n-1} \\
            \vdots & \vdots & \vdots & \ddots & \vdots \\
            1 & a_n & a_n^2 & ... & a_n^{n-1} \\
        \end{bmatrix}$
        \normalsize

        Then, det(V) = $\prod_{1 \leq i < j \leq n}$ $(a_j - a_i)$
    \end{wtheorem}

    \begin{proof}
        Let V = $V_0$. Let $V_i$ be V with the first i-1 rows and last i-1 columns
        removed. Let $W_i$ be $V_i$, but each j-th column from the last
        to the second is subtracted by $a_i$ times the (j-1)-th column.

        \hspace{0.5cm}
        \scriptsize
        $W_1$ =
        $\begin{bmatrix}
            1 & a_1-a_1 & a_1^2-a_1a_1 & ... & a_1^{n-1}-a_1^{n-2}a_1 \\
            1 & a_2-a_1 & a_2^2-a_2a_1 & ... & a_2^{n-1}-a_2^{n-2}a_1 \\
            \vdots & \vdots & \vdots & \ddots & \vdots \\
            1 & a_n-a_1 & a_n^2-a_na_1 & ... & a_n^{n-1}-a_n^{n-2}a_1 \\
        \end{bmatrix}$ =
        $\begin{bmatrix}
            1 & 0 & 0 & ... & 0 \\
            1 & a_2-a_1 & a_2(a_2-a_1) & ... & a_2^{n-2}(a_2-a_1) \\
            \vdots & \vdots & \vdots & \ddots & \vdots \\
            1 & a_n-a_1 & a_n(a_n-a_1) & ... & a_n^{n-2}(a_n-a_1) \\
        \end{bmatrix}$
        \normalsize

        By {\color{red} theorem 4.2.1}, det($W_1$) = det($V_1$) = det(V).
        Let $A_i$ be $W_i$ be the first row and column removed.
        By {\color{red} theorem 4.1.7}, for det($W_1$) across the first row,
        since only the first entry in the first row is nonzero, then
        det($W_1$) = det($A_1$).
        But for each k-th row in $A_i$, $(a_{k+i}-a_i)$ is a factor for
        k = \{1,...,n-i\}.
        After factoring out each $(a_{k+1}-a_i)$, then $A_i$ becomes $V_{i+2}$.
        
        Thus, by {\color{red} theorem 4.2.1}:

        \hspace{0.4cm}
        det($V_1$)
        = det($W_1$)
        = det($A_1$)
        = $(a_2-a_1)(a_3-a_1)...(a_n-a_1)$det($V_2$)
        = ($\prod_{i_1=2}^n (a_{i_1} - a_1)$)det($V_2$)

        Repeating the process until $V_n$:

        \hspace{0.5cm}
        det($V_1$)
        = ($\prod_{i_1=2}^n (a_{i_1} - a_1)$)det($V_2$)
        = ($\prod_{i_1=2}^n (a_{i_1} - a_1)$)
            ($\prod_{i_2=3}^n (a_{i_2} - a_2)$)det($V_3$)

        \hspace{1.9cm}
        = ...
        = ($\prod_{i_1=2}^n (a_{i_1} - a_1)$)
            ($\prod_{i_2=3}^n (a_{i_2} - a_2)$)
            ...($\prod_{i_{n-1}=n}^n (a_{i_{n-1}} - a_{n-1})$)det($V_n$)

        Since $V_n$ = $[1]$, then det($V_n$) = 1. Thus:

        \hspace{0.5cm}
        det(V) = det($V_1$)
        = ($\prod_{i_1=2}^n (a_{i_1} - a_1)$)
            ...($\prod_{i_{n-1}=n}^n (a_{i_{n-1}} - a_{n-1})$)
        = $\prod_{1 \leq i < j \leq n}$ $(a_j - a_i)$
    \end{proof}

    \vspace{0.5cm}



    \begin{wtheorem}{Eigenvectors from different Eigenspaces are
    Linearly independent}{16cm}
        Let $\lambda_1,...,\lambda_k$ $\in$ $\mathbb{R}$ be distinct
        eigenvalues of A $\in$ $M_{n \times n}(\mathbb{R})$
        with eigenvector $v_i$ $\in$ $E_{\lambda_i}$.
        Then, $v_1,...,v_k$ are linearly independent.
    \end{wtheorem}

    \begin{proof}
        Let $c_1,...,c_k$ $\in$ $\mathbb{R}$ be such that
        $c_1v_1 + ... + c_kv_k$ = 0.
        Since A$v_i$ = $\lambda_i v_i$, then for m = \{1,...,k\}:

        \hspace{0.5cm}
        $A^m v_i$
        = $A^{m-1}(\lambda_i v_i)$
        = $\lambda_i A^{m-1}(v_i)$
        = $\lambda_i A^{m-2}(\lambda_i v_i)$
        = $\lambda_i^2 A^{m-2}(v_i)$
        = ...
        = $\lambda_i^m v_i$

        Thus, the system of equations:

        \hspace{0.5cm}
        \footnotesize
        $\begin{matrix*}[l]
            0
            = A0
            = A(c_1v_1 + ... + c_kv_k)
            = c_1\lambda_1 v_1 + ... + c_k\lambda_k v_k \\
            0
            = A^20
            = A^2(c_1v_1 + ... + c_kv_k)
            = c_1\lambda_1^2 v_1 + ... + c_k\lambda_k^2 v_k \\
            \vdots \\
            0
            = A^k0
            = A^k(c_1v_1 + ... + c_kv_k)
            = c_1^k\lambda_1 v_1 + ... + c_k^k\lambda_k v_k
        \end{matrix*}$
        \normalsize

        as a matrix:

        \hspace{0.5cm}
        \scriptsize
        $0_{n \times k}$ =
        $\begin{bmatrix}
            v_1 & ... & v_k
        \end{bmatrix}
        \begin{bmatrix}
            c_1 \lambda_1 & c_1 \lambda_1^2 & \hdots & c_1 \lambda_1^k \\
            c_2 \lambda_2 & c_2 \lambda_2^2 & \hdots & c_2 \lambda_2^k \\
            \vdots & \vdots & \ddots & \vdots \\
            c_k \lambda_k & c_k \lambda_k & \hdots & c_k \lambda_k^k
        \end{bmatrix}$ =
        $\begin{bmatrix}
            v_1 & ... & v_k
        \end{bmatrix}
        \begin{bmatrix}
            c_1 & 0 & \hdots & 0 \\
            0 & c_2 & \hdots & 0 \\
            \vdots & \vdots & \ddots & \vdots \\
            0 & 0 & \hdots & c_k \\
        \end{bmatrix}
        \begin{bmatrix}
            \lambda_1 & \lambda_1^2 & \hdots & \lambda_1^k \\
            \lambda_2 & \lambda_2^2 & \hdots & \lambda_2^k \\
            \vdots & \vdots & \ddots & \vdots \\
            \lambda_k & \lambda_k & \hdots & \lambda_k^k
        \end{bmatrix}$
        = VCM
        \normalsize

        Since each i-th row has a factor of $\lambda_i$, by
        {\color{red} theorem 4.2.1} and {\color{red} 5.3.4}, then:

        \hspace{0.5cm}
        det(M) = $\lambda_1...\lambda_k
                    \prod_{1 \leq i < j \leq k}$ $(\lambda_j - \lambda_i)$

        Since each $\lambda_i$ $\not =$ 0 are distinct, then det(M) $\not =$ 0
        so by {\color{red} theorem 4.2.2}, then M is invertible.

        \hspace{0.5cm}
        $0_{n \times k}$
        = $0_{n \times k}M^{-1}$
        = VCM$M^{-1}$
        = VC
        = $\begin{bmatrix}
            c_1v_1 & ... & c_kv_k
        \end{bmatrix}$

        Since eigenvectors $v_i$ $\not =$ 0, then each $c_i$ = 0.
        Thus, $v_1,...,v_k$ are linearly independent.
    \end{proof}

    \newpage



    \begin{corollary}{Diagonalizable $M_{n \times n}(\mathbb{R})$
    $\Leftrightarrow$ $\sum$ gemu($\lambda_i$) = n}{16cm}
        Let $\lambda_1,...,\lambda_k$ $\in$ $\mathbb{R}$ be distinct
        eigenvalues of A $\in$ $M_{n \times n}(\mathbb{R})$.

        Then, A is diagonalizable if and only if
        gemu($\lambda_1$) + ... + gemu($\lambda_k$) = n.
    \end{corollary}

    \begin{proof}
        Suppose A is diagonalizable.
        By {\color{red} theorem 5.1.5}, there exists an eigenbasis
        $v_1,...,v_n$ for A.
        Since each $v_i$ $\in$ $E_{\lambda_j}$ for some j = \{1,...,k\}, then
        n $\leq$ gemu($\lambda_1$) + ... + gemu($\lambda_k$).
        By {\color{red} theorem 5.3.3}, then
        gemu($\lambda_1$) + ... + gemu($\lambda_k$) = n.

        \vspace{0.3cm}

        Suppose gemu($\lambda_1$) + ... + gemu($\lambda_k$) = n.
        Let eigenvectors $v_{i1},...,v_{im_i}$ form a basis for $E_{\lambda_i}$
        so $m_1 + ... + m_k$ = n.
        By {\color{red} theorem 5.3.5}, eigenvectors from different
        eigenspaces are linearly independent so
        $v_{11},...,v_{1m_1},v_{21},...,v_{2m_2},...,v_{k1},...,v_{km_k}$
        are linearly independent.
        By {\color{red} theorem 2.3.5},
        $v_{11},...,v_{1m_1},v_{21},...,v_{2m_2},...,v_{k1},...,v_{km_k}$
        form a basis of eigenvectors.
        By {\color{red} theorem 5.1.5}, then A is diagonalizable.
    \end{proof}

    \vspace{0.5cm}




    \begin{wtheorem}{Diagonalizable $M_{n \times n}(\mathbb{R})$
    with one Eigenvalue}{16cm}
        Let diagonalizable A $\in$ $M_{n \times n}(\mathbb{R})$ have
        only eigenvalue $\lambda$. Then:
        
        \hspace{0.5cm}
        A = $\lambda I_{n \times n}$
    \end{wtheorem}
    
    \begin{proof}
        Since A is diagonalizable, by {\color{orange} corollary 5.3.6},
        then gemu($\lambda$) = n.
        Thus, there are n linearly independent eigenvectors $v_1,...,v_n$.
        with eigenvalue $\lambda$. By {\color{red} theorem 2.3.5},
        $v_1,...,v_n$ span $\mathbb{R}^n$.
        Then for any v $\in$ $\mathbb{R}^n$, then
        v = $c_1v_1 + ... + c_nv_n$ for some $c_1,...,c_n$ $\in$ $\mathbb{R}$:

        \hspace{0.5cm}
        Av
        = A($c_1v_1 + ... + c_nv_n$)
        = $c_1\lambda v_1 + ... + c_n\lambda v_n$
        = $\lambda(c_1v_1 + ... + c_nv_n)$
        = $\lambda$v

        Thus, any v $\in$ $\mathbb{R}^n$ is an eigenvector with
        eigenvalue $\lambda$.

        \hspace{0.5cm}
        Av = $\lambda$v
        \hspace{1cm}
        $\Rightarrow$
        \hspace{1cm}
        $(A - \lambda I_{n \times n})$v = 0

        Then for v $\not =$ 0, then
        $A - \lambda I_{n \times n}$ = $0_{n \times n}$
        so A = $\lambda I_{n \times n}$.
    \end{proof}

    \vspace{0.5cm}



    \begin{example}
        Find the eigenvectors and their eigenvectors for
        A =
        \scriptsize
        $\begin{bmatrix}
            1 & 9 & 3 \\
            0 & 4 & 1 \\
            -1 & 1 & 1
        \end{bmatrix}$
        \normalsize.
        Also, find the algebraic and geometric multiplicities and det(A).
        Then, determine if A is diagonalizable.
    \end{example}

    \begin{tbox}
        0 = det($A - \lambda I_{3 \times 3}$) =
        det(
        \scriptsize
        $\begin{bmatrix}
            1-\lambda & 9 & 3 \\
            0 & 4-\lambda & 1 \\
            -1 & 1 & 1-\lambda
        \end{bmatrix}$
        \normalsize)
        = $-\lambda^3 + 6\lambda^2 - 11\lambda + 6$
        = $-(\lambda-1)(\lambda-2)(\lambda-3)$

        Thus, the eigenvalues of A are $1,2,3$ where
        almu(1) = 1, almu(2) = 1, and almu(3) = 1.

        Then, det(A) = $1^12^13^1$ = 6.
        Then, the eigenvalues for each eigenvalue:

        \hspace{0.5cm}
        For $\lambda$ = 1:
        \hspace{0.5cm}
        ker($A - I_{3 \times 3}$) =
        ker(
        \scriptsize
        $\begin{bmatrix}
            0 & 9 & 3 \\
            0 & 3 & 1 \\
            -1 & 1 & 0
        \end{bmatrix}$
        \normalsize) =
        \scriptsize
        $x_3\begin{bmatrix}
            -1 \\
            -1 \\
            3
        \end{bmatrix}$
        \normalsize
        \hspace{0.5cm}
        $\Rightarrow$
        \hspace{0.5cm}
        gemu(1) = 1

        \hspace{0.5cm}
        For $\lambda$ = 2:
        \hspace{0.5cm}
        ker($A - 2I_{3 \times 3}$) =
        ker(
        \scriptsize
        $\begin{bmatrix}
            -1 & 9 & 3 \\
            0 & 2 & 1 \\
            -1 & 1 & -1
        \end{bmatrix}$
        \normalsize) =
        \scriptsize
        $x_3\begin{bmatrix}
            -3 \\
            -1 \\
            2
        \end{bmatrix}$
        \normalsize
        \hspace{0.1cm}
        $\Rightarrow$
        \hspace{0.5cm}
        gemu(2) = 1

        \hspace{0.5cm}
        For $\lambda$ = 3:
        \hspace{0.5cm}
        ker($A - 3I_{3 \times 3}$) =
        ker(
        \scriptsize
        $\begin{bmatrix}
            -2 & 9 & 3 \\
            0 & 1 & 1 \\
            -1 & 1 & -2
        \end{bmatrix}$
        \normalsize) =
        \scriptsize
        $x_3\begin{bmatrix}
            -3 \\
            -1 \\
            1
        \end{bmatrix}$
        \normalsize
        \hspace{0.1cm}
        $\Rightarrow$
        \hspace{0.5cm}
        gemu(3) = 1

        Since gemu(1) + gemu(2) + gemu(3) = 3, then A is diagonalizable.
        
        Then, $E_1 = (-1,-1,3) +  E_2 = (-3,-1,2) + E_3 = (-3,-1,1)$
        form an eigenbasis for A where:

        \hspace{0.5cm}
        \scriptsize
        $\begin{bmatrix}
            1 & 9 & 3 \\
            0 & 4 & 1 \\
            -1 & 1 & 1
        \end{bmatrix}
        \begin{bmatrix}
            -1 & -3 & -3 \\
            -1 & -1 & -1 \\
            3 & 2 & 1
        \end{bmatrix}$ =
        $\begin{bmatrix}
            -1 & -3 & -3 \\
            -1 & -1 & -1 \\
            3 & 2 & 1
        \end{bmatrix}
        \begin{bmatrix}
            1 & 0 & 0 \\
            0 & 2 & 0 \\
            0 & 0 & 3
        \end{bmatrix}$
        \normalsize
    \end{tbox}

    \newpage



    

\subsection{ Symmetry }

    \begin{definition}{Orthogonally Diagonalizable}{16cm}
        Matrix A $\in$ $M_{n \times n}(\mathbb{R})$ is
        {\color{lblue} orthogonally diagonalizable} if there is
        an orthogonal matrix B such that $B^{-1}AB$ is diagonal

        \vspace{0.3cm}

        Since B is orthogonal, then by {\color{red} theorem 3.3.2},
        $B^{-1}$ = $B^T$ so $B^{-1}AB$ = $B^TAB$ is diagonal.

        By {\color{red} theorem 5.1.5}, then B =
        $\begin{bmatrix}
            v_1 & ... & v_n
        \end{bmatrix}$
        where $v_1,...,v_n$ is an eigenbasis for A.
        Since B is orthogonal, then by {\color{red} theorem 3.3.2},
        $v_1,...,v_n$ is orthonormal.
        Thus, if A is orthogonally diagonalizable,
        there is an orthonormal eigenbasis for A.
    \end{definition}

    \vspace{0.5cm}



    \begin{wtheorem}{Eigenvectors from different Eigenspaces are Orthogonal}{16cm}
        Let $\lambda_1,\lambda_2$ be distinct eigenvalues of
        symmetric A $\in$ $M_{n \times n}(\mathbb{R})$.

        For $x_1$ $\in$ $E_{\lambda_1}$ and $x_2$ $\in$ $E_{\lambda_2}$,
        then $x_1 \cdot x_2$ = 0
    \end{wtheorem}

    \begin{proof}
        $\lambda_1(x_1 \cdot x_2)$
        = $\lambda_1x_1 \cdot x_2$
        = $Ax_1 \cdot x_2$
        = $(Ax_1)^Tx_2$
        = $x_1^TA^Tx_2$

        \hspace{1.8cm}
        = $x_1 \cdot A^Tx_2$
        = $x_1 \cdot Ax_2$
        = $x_1 \cdot (\lambda_2x_2)$
        = $\lambda_2(x_1 \cdot x_2)$

        Since $(\lambda_1-\lambda_2)(x_1 \cdot x_2)$ = 0,
        but $\lambda_1$ $\not =$ $\lambda_2$, then $x_1 \cdot x_2$ = 0.
    \end{proof}

    \vspace{0.5cm}



    \begin{wtheorem}{Symmetric $M_{n \times n}(\mathbb{R})$ has n
    real Eigenvalues}{16cm}
        Let $\lambda_1,...\lambda_k$ $\in$ $\mathbb{R}$ be distinct
        eigenvalues of symmetric A $\in$ $M_{n \times n}(\mathbb{R})$
        with almu($\lambda_i$) = $m_i$. Then, $m_1 + ... + m_k$ = n.
    \end{wtheorem}

    \begin{proof}
        Proof requires basic properties of complex vectors
        which isn't discussed in these notes. 

        Let $p_1(\lambda)$: $\mathbb{R}$ $\rightarrow$ $\mathbb{R}$
        be the characteristic polynomial of A and
        let $p_2(\lambda)$: $\mathbb{C}$ $\rightarrow$ $\mathbb{C}$
        be the characteristic polynomial of A.
        Thus, $p_2(\lambda)$ = $p_1(\lambda)$ = det($A - \lambda I_{n \times n}$)
        if $\lambda$ $\in$ $\mathbb{R}$.

        Let $p_2(\lambda)$ = $(\lambda - \lambda_1)...(\lambda - \lambda_n)$.
        Let v = $v_1 + v_2i$ be an eigenvector with eigenvalue
        $\lambda_i$ = $a + bi$.

        \hspace{0.5cm}
        $Av_1 + iAv_2$
        = $A(v_1+v_2i)$
        = $Av$
        = $\lambda_i v$
        = $(a+bi)(v_1 + v_2i)$
        = $(av_1 - bv_2) + (av_2 + bv_1)i$

        Since:

        \hspace{0.5cm}
        $A\overline{v}$
        = $A(v_1 - v_2i)$
        = $Av_1 - iAv_2$
        \hspace{1cm}
        $\overline{\lambda_i} \overline{v}$
        = $(a-bi)(v_1 - v_2i)$
        = $(av_1 - bv_2) - (av_2 + bv_1)i$

        then $A\overline{v}$ = $\overline{\lambda_i} \overline{v}$.
        Thus, $\overline{\lambda_i}$ is an eigenvalue of A with eigenvector
        $\overline{v}$. Then:

        \hspace{0.5cm}
        $\overline{v}^T Av$
        = $\overline{v}^T (\lambda_i v)$
        = $\lambda_i(\overline{v}^T v)$
        = $\lambda_i |v|^2$
        \hspace{0.5cm}
        $\overline{v}^T Av$
        = $\overline{v}^T \overline{A^T} v$
        = $\overline{Av}^T v$
        = $\overline{\lambda_i v}^T v$
        = $\overline{\lambda_i} \overline{v}^T v$
        = $\overline{\lambda_i} |v|^2$

        Since $(\lambda_i - \overline{\lambda_i})|v|^2$ = 0,
        but $|v|$ $>$ 0 since v $\not =$ 0, then
        $a+bi$ = $\lambda_i$ = $\overline{\lambda_i}$ = $a-bi$.
        Thus, b = 0 so all eigenvalues are real.
        Thus, any $p_1(\lambda)$ = $p_2(\lambda)$
        = $(\lambda - \lambda_1)...(\lambda - \lambda_n)$
        so A has n real eigenvalues when including their algebraic multiplicity.
    \end{proof}

    \newpage



    \begin{wtheorem}{Spectral Theorem: Symmetric matrices have an
    orthonormal eigenbasis}{16cm}
        Matrix A $\in$ $M_{n \times n}(\mathbb{R})$ is orthogonally
        diagonalizable if and only if A is symmetric
    \end{wtheorem}

    \begin{proof}
        Suppose A is orthogonally diagonalizable.
        Thus, A has an $\mathcal{B}$ = $v_1,...,v_n$ $\in$ $\mathbb{R}^n$
        is an orthonormal eigenbasis with eigenvalues $\lambda_1,...,\lambda_n$
        $\in$ $\mathbb{R}$.
        Since A is diagonalizable, then by {\color{red} theorem 5.1.5},
        for B =
        $\begin{bmatrix}
            v_1 & ... & v_n
        \end{bmatrix}$ and $A_{\mathcal{B}}$ =
        $\begin{bmatrix}
            \lambda_1e_1 & ... & \lambda_ne_n
        \end{bmatrix}$:

        \hspace{0.5cm}
        AB = B$A_{\mathcal{B}}$

        By {\color{red} theorem 3.3.2}, B is invertible where $B^{-1}$ = $B^T$.
        Thus, A = $BA_{\mathcal{B}}B^{-1}$ = $BA_{\mathcal{B}}B^T$. Then:

        \hspace{0.5cm}
        $A^T$
        = $(BA_{\mathcal{B}}B^T)^T$
        = $(B^T)^T A_{\mathcal{B}}^T B^T$
        = $B A_{\mathcal{B}} B^T$
        = A

        Thus, A is symmetric.

        \vspace{0.3cm}

        Suppose A is symmetric.
        If n = 1, then let B = $[1]$ which is orthogonal since
        $B^TB$ = $I_{1 \times 1}$ by {\color{red} theorem 3.3.2}.
        Then, $B^{-1}AB$ = $[1][a][1]$ = $[a]$ is diagonal.

        Suppose for k $\leq$ n, then any symmetric
        $M_{(k-1) \times (k-1)}(\mathbb{R})$ is orthogonally diagonalizable.
        Then for symmetric A $\in$ $M_{k \times k}(\mathbb{R})$,
        by {\color{red} theorem 5.4.3}, there is an eigenvalue $\lambda$
        with eigenvector $v_1$ such that $|v|$ = 1.
        From $v_1$, let $v_1,u_2,...,u_k$ be a basis for $\mathbb{R}^k$.
        Apply {\color{red} theorem 3.2.9} to get an orthonormal basis
        $v_1,...,v_k$. Let B =
        $\begin{bmatrix}
            v_1 & ... & v_k
        \end{bmatrix}$ so B is orthogonal by {\color{red} 3.3.2}
        so $B^T$ = $B^{-1}$.
        
        Note $B^TABe_1$
        = $B^TAv_1$
        = $B^T (\lambda v_1)$
        = $\lambda (B^Tv_1)$
        = $\lambda e_1$.
        Also, $(B^TAB)$ = $B^TA^T(B^T)^T$ = $B^TAB$ so $B^TAB$ is symmetric.
        Thus,
        $B^TAB$ =
        $\begin{bmatrix}
            \lambda & 0_{1 \times (k-1)} \\
            0_{(k-1) \times 1} & B^*_{(k-1) \times (k-1)}
        \end{bmatrix}$.

        Since $B^*$ $\in$ $M_{(k-1) \times (k-1)}(\mathbb{R})$,
        then $B^*$ is orthogonally diagonalizable so there is an orthogonal
        C $\in$ $M_{(k-1) \times (k-1)}(\mathbb{R})$ such that
        D = $C^{-1}B^*C$ is diagonal.
        Let X = $\begin{bmatrix}
            1 & 0_{1 \times (k-1)} \\
            0_{(k-1) \times 1} & C
        \end{bmatrix}$. Then:

        \hspace{0.5cm}
        $X^T B^T A B X$ =
        $\begin{bmatrix}
            1 & 0_{1 \times (k-1)} \\
            0_{(k-1) \times 1} & C^T
        \end{bmatrix}
        \begin{bmatrix}
            \lambda & 0_{1 \times (k-1)} \\
            0_{(k-1) \times 1} & B^*
        \end{bmatrix}
        \begin{bmatrix}
            1 & 0_{1 \times (k-1)} \\
            0_{(k-1) \times 1} & C
        \end{bmatrix}$

        \hspace{2.7cm}
        =
        $\begin{bmatrix}
            \lambda & 0_{1 \times (k-1)} \\
            0_{(k-1) \times 1} & C^TB^*C
        \end{bmatrix}$ =
        $\begin{bmatrix}
            \lambda & 0_{1 \times (k-1)} \\
            0_{(k-1) \times 1} & D
        \end{bmatrix}$

        Since $X^T B^T A B X$ = $(BX)^T A (BX)$ is diagonal
        where BX is orthogonal since B,X are orthogonal
        by {\color{orange} corollary 3.3.4}, then
        A $\in$ $M_{k \times k}(\mathbb{R})$ is orthogonally diagonalizable.
        Thus, by proof by induction, A $\in$ $M_{n \times n}(\mathbb{R})$
        is orthogonally diagonalizable.
    \end{proof}

    \newpage



    \begin{example}
        Find an orthonormal eigenbasis for A =
        $\begin{bmatrix}
            1 & 1 & 1 \\
            1 & 1 & 1 \\
            1 & 1 & 1
        \end{bmatrix}$
    \end{example}

    \begin{tbox}
        0 = det($A - \lambda I_{3 \times 3}$) =
        det(
        \scriptsize
        $\begin{bmatrix}
            1-\lambda & 1 & 1 \\
            1 & 1-\lambda & 1 \\
            1 & 1 & 1-\lambda
        \end{bmatrix}$
        \normalsize)
        = $-\lambda^3 + 3\lambda^2$
        = $-\lambda^2(\lambda-3)$

        Thus, the eigenvalues of A are $0,3$ where
        almu(0) = 2 and almu(3) = 1.

        Then, det(A) = $(-2)^12^13^1$ = -6.
        Then, the eigenvalues for each eigenvalue:

        \hspace{0.5cm}
        For $\lambda$ = 0:
        \hspace{0.5cm}
        ker($A - 0I_{3 \times 3}$) =
        ker(
        \scriptsize
        $\begin{bmatrix}
            1 & 1 & 1 \\
            1 & 1 & 1 \\
            1 & 1 & 1
        \end{bmatrix}$
        \normalsize) =
        \scriptsize
        $x_2\begin{bmatrix}
            -1 \\
            1 \\
            0
        \end{bmatrix} +
        x_3\begin{bmatrix}
            -1 \\
            0 \\
            1
        \end{bmatrix}$
        \normalsize
        \hspace{0.1cm}
        $\Rightarrow$
        \hspace{0.5cm}
        gemu(0) = 2

        \hspace{0.5cm}
        For $\lambda$ = 3:
        \hspace{0.45cm}
        ker($A - 3I_{3 \times 3}$) =
        ker(
        \scriptsize
        $\begin{bmatrix}
            -2 & 1 & 1 \\
            1 & -2 & 1 \\
            1 & 1 & -2
        \end{bmatrix}$
        \normalsize) =
        \scriptsize
        $x_3\begin{bmatrix}
            1 \\
            1 \\
            1
        \end{bmatrix}$
        \normalsize
        \hspace{1.2cm}
        $\Rightarrow$
        \hspace{0.5cm}
        gemu(3) = 1
        
        Then, $E_0$ = $(-1,1,0), (-1,0,1)$ + $e_3$ = $(1,1,1)$
        form an eigenbasis for A.

        To find an orthonormal eigenbasis, apply the Gram-Schmidt Process
        to each eigenspace.

        \hspace{0.5cm}
        \scriptsize
        $\begin{bmatrix}
            -1 \\
            1 \\
            0
        \end{bmatrix}$

        \hspace{0.5cm}
        $\begin{bmatrix}
            -1 \\
            0 \\
            1
        \end{bmatrix}
        - \frac{1}{(-1)^2 + 1^2 + 0^2}
        (\begin{bmatrix}
            -1 \\
            0 \\
            1
        \end{bmatrix} \cdot
        \begin{bmatrix}
            -1 \\
            1 \\
            0
        \end{bmatrix})
        \begin{bmatrix}
            -1 \\
            1 \\
            0
        \end{bmatrix}$ =
        $\begin{bmatrix}
            -1 \\
            0 \\
            1
        \end{bmatrix}
        - \frac{1}{2}
        \begin{bmatrix}
            -1 \\
            1 \\
            0
        \end{bmatrix}$ =
        $\begin{bmatrix}
            -\frac{1}{2} \\
            -\frac{1}{2} \\
            1
        \end{bmatrix}$
        \normalsize

        Thus,
        $(-\frac{1}{\sqrt{2}},\frac{1}{\sqrt{2}},0),
        (-\frac{1}{\sqrt{6}},-\frac{1}{\sqrt{6}},\frac{2}{\sqrt{6}}),
        (\frac{1}{\sqrt{3}},\frac{1}{\sqrt{3}},\frac{1}{\sqrt{3}})$
        form an orthonormal eigenbasis for A where

        B =
        \scriptsize
        $\begin{bmatrix}
            -\frac{1}{\sqrt{2}} & -\frac{1}{\sqrt{6}} & \frac{1}{\sqrt{3}} \\
            \frac{1}{\sqrt{2}} & -\frac{1}{\sqrt{6}} & \frac{1}{\sqrt{3}} \\
            0 & \frac{2}{\sqrt{6}} & \frac{1}{\sqrt{3}}
        \end{bmatrix}$
        \normalsize
        is orthogonal so $B^{-1}$ = $B^T$ such that:

        \hspace{0.5cm}
        \scriptsize
        $\begin{bmatrix}
            1 & 1 & 1 \\
            1 & 1 & 1 \\
            1 & 1 & 1 \\
        \end{bmatrix}
        \begin{bmatrix}
            -\frac{1}{\sqrt{2}} & -\frac{1}{\sqrt{6}} & \frac{1}{\sqrt{3}} \\
            \frac{1}{\sqrt{2}} & -\frac{1}{\sqrt{6}} & \frac{1}{\sqrt{3}} \\
            0 & \frac{2}{\sqrt{6}} & \frac{1}{\sqrt{3}}
        \end{bmatrix}$ =
        $\begin{bmatrix}
            -\frac{1}{\sqrt{2}} & -\frac{1}{\sqrt{6}} & \frac{1}{\sqrt{3}} \\
            \frac{1}{\sqrt{2}} & -\frac{1}{\sqrt{6}} & \frac{1}{\sqrt{3}} \\
            0 & \frac{2}{\sqrt{6}} & \frac{1}{\sqrt{3}}
        \end{bmatrix}
        \begin{bmatrix}
            0 & 0 & 0 \\
            0 & 0 & 0 \\
            0 & 0 & 3
        \end{bmatrix}$
        \normalsize
    \end{tbox}

    \newpage





\subsection{ Quadratic Forms }

    \begin{definition}{Quadratic Form}{16cm}
        f: $\mathbb{R}^n$ $\rightarrow$ $\mathbb{R}$
        is a {\color{lblue} quadratic form} if there are
        $a_{ij}$ $\in$ $\mathbb{R}$ such that:

        \hspace{0.5cm}
        f(x) = $\sum_{i=1}^n \sum_{j=1}^n$ $a_{ij}x_ix_j$
    \end{definition}

    \vspace{0.5cm}



    \begin{wtheorem}{Matrix Form of the Quadratic Form}{16cm}
        Let f: $\mathbb{R}^n$ $\rightarrow$ $\mathbb{R}$
        be a quadratic form. Then, there is a unique symmetric
        A $\in$ $M_{n \times n}(\mathbb{R})$:

        \hspace{0.5cm}
        f(x) = x $\cdot$ Ax

        If B $\in$ $M_{n \times n}(\mathbb{R})$ satisfies
        f(x) = x $\cdot$ Bx, then A = $\frac{1}{2}(B^T + B)$.
    \end{wtheorem}

    \begin{proof}
        Let A =
        $\begin{bmatrix}
            a_{11} & \hdots & a_{1n} \\
            \vdots & \ddots & \vdots \\
            a_{n1} & \hdots & a_{nn}
        \end{bmatrix}$. Since each x = $\sum_{k=1}^n x_ke_k$, then:

        \hspace{0.5cm}
        f(x)
        = $\sum_{i=1}^n \sum_{j=1}^n$ $a_{ij}x_ix_j$
        = $\sum_{i=1}^n \sum_{j=1}^n$ $(e_i \cdot Ae_j)x_ix_j$
        = $\sum_{i=1}^n \sum_{j=1}^n$ $(e_i \cdot A(x_je_j))x_i$

        \hspace{1.3cm}
        = $\sum_{i=1}^n$ $(e_i \cdot Ax)x_i$
        = $\sum_{i=1}^n$ $x_ie_i \cdot Ax$
        = $x \cdot Ax$

        Let B $\in$ $M_{n \times n}(\mathbb{R})$ satisfy f(x) = x $\cdot$ Bx.

        \hspace{0.5cm}
        f(x)
        = x $\cdot$ Bx
        = $\frac{1}{2}(Bx \cdot x + x \cdot Bx)$
        = $\frac{1}{2}(x \cdot B^Tx + x \cdot Bx)$
        = $\frac{1}{2}(x \cdot (B^T + B)x)$
        = $x \cdot \frac{1}{2}(B^T + B)x$

        Thus, A = $\frac{1}{2}(B^T + B)$.
        Since $(\frac{1}{2}(B^T + B))^T$
        = $\frac{1}{2}((B^T)^T + B^T)$
        = $\frac{1}{2}(B + B^T)$
        = $\frac{1}{2}(B^T + B)$, then A is symmetric.
        Let symmetric C $\in$ $M_{n \times n}(\mathbb{R})$
        also satisfy f(x) = x $\cdot$ Cx.
        Let symmetric D = A - C where for any x $\in$ $\mathbb{R}^n$:

        \hspace{0.5cm}
        $x \cdot Dx$
        = $(x \cdot (A-C)x)$
        = $(x \cdot Ax) - (x \cdot Cx)$
        = f(x) - f(x)
        = 0.
        
        By {\color{red} theorem 5.4.4}, then D has an orthonormal eigenbasis
        $v_1,...,v_n$ with eigenvalues $\lambda_1,...,\lambda_n$.
        Then for i = \{1,...,n\}

        \hspace{0.5cm}
        0 = $v_i \cdot Dv_i$
        = $v_i \cdot \lambda_i v_i$
        = $\lambda_i |v_i|^2$

        Since eigenvector $v_i$ $\not =$ 0 so $|v_i|$ $>$ 0, then
        all $\lambda_i$ = 0. By {\color{red} theorem 5.3.7},
        D = $0 I_{n \times n}$ = $0_{n \times n}$ so A = C.
        Thus, A is unique.
    \end{proof}

    \vspace{0.5cm}



    \begin{wtheorem}{Eigenvalue Form of the Quadratic Form}{16cm}
        For f: $\mathbb{R}^n$ $\rightarrow$ $\mathbb{R}$,
        let f(x) = x $\cdot$ Ax. Then, there is an orthonormal eigenbasis
        $v_1,...,v_n$ for A with eigenvalues $\lambda_1,...,\lambda_n$ such that:

        \hspace{0.5cm}
        f(x) = $\lambda_1(x \cdot v_1)^2 + ... + \lambda_n(x \cdot v_n)^2$

        For any x $\in$ $\mathbb{R}^n$
        such that x = $c_1v_1 + ... + c_nv_n$ where each
        $c_i$ = $(x \cdot v_i)$ $\in$ $\mathbb{R}$, then:

        \hspace{0.5cm}
        f(x) = $\lambda_1c_1^2 + ... + \lambda_nc_n^2$
    \end{wtheorem}

    \begin{proof}
        By {\color{red} theorem 5.4.4}, A has an orthonormal eigenbasis
        $v_1,...,v_n$ with eigenvalues $\lambda_1,...,\lambda_n$.
        
        For x $\in$ $\mathbb{R}^n$, by {\color{red} theorem 3.2.6},
        then x = $(x \cdot v_1)v_1 + ... + (x \cdot v_n)v_n$.
        Since $v_1,...,v_n$ is orthonormal, then $v_i \cdot v_i$ = 1
        and $v_i \cdot v_j$ = 0 for i $\not =$ j. Thus:

        \hspace{0.5cm}
        f(x) = x $\cdot$ Ax
        = ($(x \cdot v_1)v_1 + ... + (x \cdot v_n)v_n$)
            $\cdot$ A($(x \cdot v_1)v_1 + ... + (x \cdot v_n)v_n$)

        \hspace{1.3cm}
        = ($(x \cdot v_1)v_1 + ... + (x \cdot v_n)v_n$)
            $\cdot$ ($(x \cdot v_1)\lambda_1v_1 +...+ (x \cdot v_n)\lambda_nv_n$)

        \hspace{1.3cm}
        = $\lambda_1(x \cdot v_1)^2 + ... + \lambda_n(x \cdot v_n)^2$

        For x $\in$ $\mathbb{R}^n$,
        since $v_1,...,v_n$ is an orthonormal basis,
        by {\color{red} theorem 3.2.6}, then
        
        x = $c_1v_1 + ... + c_nv_n$ where each
        $c_i$ = $(x \cdot v_i)$ $\in$ $\mathbb{R}$. Thus:

        \hspace{0.5cm}
        f(x)
        = $\lambda_1(x \cdot v_1)^2 + ... + \lambda_n(x \cdot v_n)^2$

        \hspace{1.3cm}
        = $\lambda_1((c_1v_1 + ... + c_nv_n) \cdot v_1)^2
            + ... + \lambda_n((c_1v_1 + ... + c_nv_n) \cdot v_n)^2$

        \hspace{1.3cm}
        = $\lambda_1c_1^2 + ... + \lambda_nc_n^2$
    \end{proof}

    \newpage



    \begin{definition}{Definiteness}{16cm}
        For f: $\mathbb{R}^n$ $\rightarrow$ $\mathbb{R}$,
        let f(x) = x $\cdot$ Ax. Then:

        \begin{enumerate}[label=(\alph*), leftmargin=1cm, itemsep=0.1cm]
            \item f is {\color{lgreen} positive definite} if all
                eigenvalues of A are positive

            \item f is {\color{lgreen} negative definite} if all
                eigenvalues of A are negative

            \item f is {\color{lgreen} indefinite} if there is at least one
                positive and negative eigenvalue
        \end{enumerate}
    \end{definition}

    \vspace{0.5cm}



    \begin{example}
        Find the shape of $-3x^2 - 4xy - 6y^2$ = -1.
    \end{example}

    \begin{tbox}
        Since
        $-3x^2 - 4xy - 6y^2$ = -1
        \hspace{0.5cm}
        $\Leftrightarrow$
        \hspace{0.5cm}
        \scriptsize
        $\begin{bmatrix}
            x \\
            y
        \end{bmatrix}
        \cdot
        \begin{bmatrix}
            -3 & -4 \\
            0 & -6
        \end{bmatrix}
        \begin{bmatrix}
            x \\
            y
        \end{bmatrix}$
        \normalsize
        = -1,
        then:

        \hspace{0.5cm}
        A =
        \scriptsize
        $\frac{1}{2}(\begin{bmatrix}
            -3 & -4 \\
            0 & -6
        \end{bmatrix}^T +
        \begin{bmatrix}
            -3 & -4 \\
            0 & -6
        \end{bmatrix})$ =
        $\begin{bmatrix}
            -3 & -2 \\
            -2 & -6
        \end{bmatrix}$
        \normalsize

        Thus,
        $-3x^2 - 4xy - 6y^2$ = -1
        \hspace{0.5cm}
        $\Leftrightarrow$
        \hspace{0.5cm}
        \scriptsize
        $\begin{bmatrix}
            x \\
            y
        \end{bmatrix}
        \cdot
        \begin{bmatrix}
            -3 & -2 \\
            -2 & -6
        \end{bmatrix}
        \begin{bmatrix}
            x \\
            y
        \end{bmatrix}$
        \normalsize
        = -1.
        To find an orthonormal eigenbasis:

        \hspace{0.5cm}
        0 =
        det(
        \scriptsize
        $\begin{bmatrix}
            -3-\lambda & -2 \\
            -2 & -6-\lambda
        \end{bmatrix}$
        \normalsize)
        = $\lambda^2 + 9\lambda + 14$
        = $(\lambda+2)(\lambda+7)$

        Thus, the eigenvalues for A are -2 and -7.

        \hspace{0.5cm}
        For $\lambda$ = -2:
        \hspace{0.5cm}
        ker($A + 2I_{2 \times 2}$) =
        ker(
        \scriptsize
        $\begin{bmatrix}
            -1 & -2 \\
            -2 & -4
        \end{bmatrix}$
        \normalsize) =
        \scriptsize
        $y\begin{bmatrix}
            -2 \\
            1
        \end{bmatrix}$
        \normalsize

        \hspace{0.5cm}
        For $\lambda$ = -7:
        \hspace{0.5cm}
        ker($A + 7I_{2 \times 2}$) =
        ker(
        \scriptsize
        $\begin{bmatrix}
            4 & -2 \\
            -2 & 1
        \end{bmatrix}$
        \normalsize) =
        \scriptsize
        $y\begin{bmatrix}
            1 \\
            2
        \end{bmatrix}$
        \normalsize

        Thus,
        $E_{-2} = (-\frac{2}{\sqrt{5}},\frac{1}{\sqrt{5}})
        + E_{-7} = (\frac{1}{\sqrt{5}},\frac{2}{\sqrt{5}})$
        form an orthonormal eigenbasis for A. Thus:

        \hspace{0.5cm}
        $-3x^2 - 4xy - 6y^2$ = -1
        \hspace{0.5cm}
        $\Leftrightarrow$
        \hspace{0.5cm}
        $-2c_1^2 + -7c_2^2$ = -1
        \hspace{0.5cm}
        $\Rightarrow$
        \hspace{0.5cm}
        $2c_1^2 + 7c_2^2$ = 1
        
        Thus, $-3x^2 - 4xy - 6y^2$ = -1 is an ellipse
        with vertices $(-\frac{1}{\sqrt{2}},0)$ and $(\frac{1}{\sqrt{2}},0)$
        along the $(-\frac{2}{\sqrt{5}},\frac{1}{\sqrt{5}})$ axis and
        $(0,-\frac{1}{\sqrt{7}})$ and $(0,\frac{1}{\sqrt{7}})$ along
        the $(\frac{1}{\sqrt{5}},\frac{2}{\sqrt{5}})$ axis.
        So, on the regular xy-plane, each vertice:

        \hspace{0.5cm}
        $(-\frac{1}{\sqrt{2}},0)$
        \hspace{0.5cm}
        $\Rightarrow$
        \hspace{0.5cm}
        $(-\frac{1}{\sqrt{2}})(-\frac{2}{\sqrt{5}},\frac{1}{\sqrt{5}})
            + (0)(\frac{1}{\sqrt{5}},\frac{2}{\sqrt{5}})$
        = $(\frac{2}{\sqrt{10}},-\frac{1}{\sqrt{10}})$

        \hspace{0.5cm}
        $(\frac{1}{\sqrt{2}},0)$
        \hspace{0.8cm}
        $\Rightarrow$
        \hspace{0.5cm}
        $(\frac{1}{\sqrt{2}})(-\frac{2}{\sqrt{5}},\frac{1}{\sqrt{5}})
            + (0)(\frac{1}{\sqrt{5}},\frac{2}{\sqrt{5}})$
        = $(-\frac{2}{\sqrt{10}},\frac{1}{\sqrt{10}})$

        \hspace{0.5cm}
        $(0,-\frac{1}{\sqrt{7}})$
        \hspace{0.5cm}
        $\Rightarrow$
        \hspace{0.5cm}
        $(0)(-\frac{2}{\sqrt{5}},\frac{1}{\sqrt{5}})
            + (-\frac{1}{\sqrt{7}})(\frac{1}{\sqrt{5}},\frac{2}{\sqrt{5}})$
        = $(-\frac{1}{\sqrt{35}},-\frac{2}{\sqrt{35}})$

        \hspace{0.5cm}
        $(0,\frac{1}{\sqrt{7}})$
        \hspace{0.8cm}
        $\Rightarrow$
        \hspace{0.5cm}
        $(0)(-\frac{2}{\sqrt{5}},\frac{1}{\sqrt{5}})
            + (\frac{1}{\sqrt{7}})(\frac{1}{\sqrt{5}},\frac{2}{\sqrt{5}})$
        = $(\frac{1}{\sqrt{35}},\frac{2}{\sqrt{35}})$

        Since the eigenvalues of A are all negative,
        then $-3x^2 - 4xy - 6y^2$ = -1 is negative definite.
    \end{tbox}




