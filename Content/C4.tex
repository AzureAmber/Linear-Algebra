\newpage

\section[Day 4: Determinants]{ Determinants }

\subsection{ Determinant Function }

    \begin{definition}{Determinant Function}{16cm}
        Function D: $M_{n \times n}(\mathbb{R})$ $\rightarrow$ $\mathbb{R}$
        is a {\color{lblue} determinant function} if
        for any A $\in$ $M_{n \times n}(\mathbb{R})$
        such that A =
        $\begin{bmatrix}
            A_1 & ... & A_n
        \end{bmatrix}$, then:

        \begin{enumerate}[label=(\alph*), leftmargin=1cm, itemsep=0.1cm]
            \item {\color{lgreen} Multilinearity}:
                For any j $\in$ \{1,...,n\}, then
                T: $\mathbb{R}^n$ $\rightarrow$ $\mathbb{R}$:
                
                \hspace{0.5cm}
                T(x) = D($A_1,...,A_{i-1},x,A_{i+1},...,A_n$)

                is linear. So, for x,y $\in$ $\mathbb{R}^n$
                and c $\in$ $\mathbb{R}$:

                \begin{itemize}
                    \item T(x+y) = T(x) + T(y)
                    
                    \item T(cx) = cT(x)
                \end{itemize}

            \item {\color{lgreen} Alternating}:
                If B $\in$ $M_{n \times n}(\mathbb{R})$ is obtained by
                swapping two columns of A, then:

                \hspace{0.5cm}
                D(B) = -D(A)

                Thus, if two columns are equal, then
                by swapping those columns, B = A.

                \hspace{0.5cm}
                D(A) = D(B) = -D(A)
                \hspace{0.5cm}
                $\Rightarrow$
                \hspace{0.5cm}
                D(A) = 0

            \item {\color{lgreen} Identity}
            
                \hspace{0.5cm}
                D($I_{n \times n}$) = 1
        \end{enumerate}

        Since each $A_i$ = $A_{i1}e_1 + ... + A_{in}e_n$, then
        by multilinearity:

        \hspace{0.5cm}
        D(A)
        = D($A_1,A_2,...,A_n$)

        \hspace{1.6cm}
        = $\sum_{i_1=1}^n$
            $A_{1{i_1}}$D($e_{i_1},A_2,...,A_n$)

        \hspace{1.6cm}
        = $\sum_{i_1=1}^n \sum_{i_2=1}^n$
            $A_{1{i_1}}A_{2{i_2}}$D($e_{i_1},e_{i_2},...,A_n$)

        \hspace{1.6cm}
        $\vdots$

        \hspace{1.6cm}
        = $\sum_{i_1=1}^n ... \sum_{i_n=1}^n$
            $A_{1{i_1}}...A_{n{i_n}}$D($e_{i_1},e_{i_2},...,e_{i_n}$)

        By alternating, if $i_{k_1}$ = $i_{k_2}$ for $k_1,k_3$ $\in$ \{1,...,n\},
        then D($e_{i_1},e_{i_2},...,e_{i_n}$) = 0. Thus:

        \hspace{0.5cm}
        D(A) = $\sum_{\{i_1,...,i_n\} = \{1,...,n\}}$
                $A_{1{i_1}}...A_{n{i_n}}$D($e_{i_1},e_{i_2},...,e_{i_n}$)

        where each $i_k$ is unique from \{1,...,n\}.
        Since each $A_{k{i_k}}$ for k $\in$ \{1,...,n\} are different columns
        and each $i_k$ is unique, then each $A_{k{i_k}}$ is from a unique
        row and column from A:

        \hspace{0.5cm}
        A =
        $\begin{bmatrix}
            A_1 & ... & A_n
        \end{bmatrix}$ =
        $\begin{bmatrix}
            a_{11} & a_{12} & \hdots & a_{1n} \\
            a_{21} & a_{22} & \hdots & a_{2n} \\
            \vdots & \vdots & \ddots & \vdots \\
            a_{n1} & a_{n2} & \hdots & a_{nn}
        \end{bmatrix}$
        \hspace{1cm}
        $\Rightarrow$
        \hspace{1cm}
        $A_{k{i_k}}$ = $a_{ij}$

        Since $(e_{i_1},e_{i_2},...,e_{i_n})$
        is a rearrangement of $(e_1,...,e_n)$, then by alternating:

        \hspace{0.5cm}
        D($e_{i_1},e_{i_2},...,e_{i_n}$)
        = $(-1)^S$D($e_1,...,e_n$)
        = $(-1)^S$D($I_{n \times n}$)
        = $(-1)^S$

        where S is the number of swaps to turn
        $(e_{i_1},e_{i_2},...,e_{i_n})$ into $(e_1,...,e_n)$.
        So, D(A) is unique:

        \hspace{0.5cm}
        D(A) = $\sum_{\{i_1,...,i_n\} = \{1,...,n\}}$
                $a_{1i_1}...a_{ni_n}(-1)^S$
    \end{definition}

    \newpage



    \begin{definition}{Inversions}{16cm}
        A $n \times n$ {\color{lblue} pattern} P = \{$(i_1,j_1),...,(i_n,j_n)$\}
        where \{$i_1,...,i_n$\},\{$j_1,...,j_n$\} are rearrangements of
        \{1,...,n\}.
        Then, $(i_{k_1},j_{k_1})$ and $(i_{k_2},j_{k_2})$ where
        $j_{k_1}$ $<$ $j_{k_2}$ is an {\color{lblue} inversion}
        if $i_{k_1}$ $>$ $i_{k_2}$.

        Then, the {\color{lblue} signature} of P:

        \hspace{0.5cm}
        sgn(P) = $(-1)^I$
        \hspace{1cm}
        where I is the total number of inversions
    \end{definition}

    \vspace{0.5cm}



    \begin{wtheorem}{D($e_{i_1},e_{i_2},...,e_{i_n}$)
    = sgn($e_{i_1},e_{i_2},...,e_{i_n}$)}{16cm}
        Let $e_{i_1},e_{i_2},...,e_{i_n}$ $\in$ $\mathbb{R}^n$
        be a rearrangement of $e_1,...,e_n$ $\in$ $\mathbb{R}^n$.
        Let S be the number of swaps to turn $e_{i_1},e_{i_2},...,e_{i_n}$
        into $e_1,...,e_n$ and I be the number of inversions
        in $(1,i_1),...,(n,i_n)$.
        Then, S = I so:

        \hspace{0.5cm}
        D($e_{i_1},e_{i_2},...,e_{i_n}$)
        = $(-1)^S$
        = $(-1)^I$
        = sgn($e_{i_1},e_{i_2},...,e_{i_n}$)
    \end{wtheorem}

    \begin{proof}
        Note D($e_{i_1},e_{i_2},...,e_{i_n}$) = $(-1)^S$
        and sgn($e_{i_1},e_{i_2},...,e_{i_n}$) = $(-1)^I$.
        Suppose n = 2. Then, either:

        \hspace{0.5cm}
        D($\begin{bmatrix}
                1 \\
                0
            \end{bmatrix},
            \begin{bmatrix}
                0 \\
                1
            \end{bmatrix}$):
        \hspace{1cm}
        0 swaps , 0 inversions

        \hspace{0.5cm}
        D($\begin{bmatrix}
                0 \\
                1
            \end{bmatrix},
            \begin{bmatrix}
                0 \\
                1
            \end{bmatrix}$):
        \hspace{1cm}
        1 swap , 1 inversion

        Thus, when n = 2, then S = I so
        D($e_{i_1},e_{i_2},...,e_{i_n}$) = $(-1)^S$
        = $(-1)^I$ = sgn($e_{i_1},e_{i_2},...,e_{i_n}$).

        For k $\leq$ n, let $e_{i_1},e_{i_2},...,e_{i_{k-1}}$
        have the same number of swaps as for inversions.

        Then, for $e_{i_1},e_{i_2},...,e_{i_{k-1}},e_{i_k}$,
        the only possible new inversions compared to
        $e_{i_1},e_{i_2},...,e_{i_{k-1}}$ are $e_{i_k}$
        with each of $e_{i_1},e_{i_2},...,e_{i_{k-1}}$.
        Also, note the only new swaps are also $e_{i_k}$
        with each of $e_{i_1},e_{i_2},...,e_{i_{k-1}}$
        since if $e_{i_1},e_{i_2},...,e_{i_{k-1}}$ is swapped into
        $e_{i_1^*},e_{i_2^*},...,e_{i_{k-1^*}}$ which has no inversions,
        then $e_{i_k}$ can swap with $e_{i_{k-1^*}}$ first if needed,
        then $e_{i_{k-2^*}}$ second if needed, and etc. Thus, each
        $e_{i_1},e_{i_2},...,e_{i_{k-1}}$ does not have to swap with one another,
        the only swaps are $e_{i_k}$ with $e_{i_1},e_{i_2},...,e_{i_{k-1}}$,
        and $e_{i_1^*},e_{i_2^*},...,e_{i_{k-1^*}},e_{i_{k^*}}$
        can be reached with no inversions.

        But, since D($e_{i_k},e_{i_j}$) = sgn($e_{i_k},e_{i_j}$)
        for j = \{1,...,k-1\}, then the total new inversions
        is the same as the total new swaps.
        Thus, by proof by induction, S = I so:
        
        \hspace{0.5cm}
        D($e_{i_1},e_{i_2},...,e_{i_n}$)
        = $(-1)^S$
        = $(-1)^I$
        = sgn($e_{i_1},e_{i_2},...,e_{i_n}$)
    \end{proof}

    \vspace{0.5cm}



    \begin{corollary}{Determinant function redefined}{16cm}
        For determinant function D: $M_{n \times n}(\mathbb{R})$
        $\rightarrow$ $\mathbb{R}$, then for any
        A $\in$ $M_{n \times n}(\mathbb{R})$:

        \hspace{0.5cm}
        D(A) = $\sum_{\{i_1,...,i_n\} = \{1,...,n\}}$
                $a_{1i_1}...a_{ni_n}(-1)^I$

        where I is the number of inversions
        in $(1,i_1),...,(n,i_n)$
    \end{corollary}

    \begin{proof}
        D(A) =
        $\sum_{\{i_1,...,i_n\} = \{1,...,n\}}$
            $a_{1i_1}...a_{ni_n}(-1)^S$
        where S is the number of swaps to turn 

        $(e_{i_1},e_{i_2},...,e_{i_n})$ into $(e_1,...,e_n)$.
        Since D($e_{i_1},e_{i_2},...,e_{i_n}$) = $(-1)^S$,
        then by {\color{red} theorem 4.1.3},
        $(-1)^S$ = sgn($e_{i_1},e_{i_2},...,e_{i_n}$) = $(-1)^I$
        where I is the number of inversions
        in $(1,i_1),...,(n,i_n)$. Thus:

        \hspace{0.5cm}
        D(A)
        = $\sum_{\{i_1,...,i_n\} = \{1,...,n\}}$
            $a_{1i_1}...a_{ni_n}(-1)^S$
        = $\sum_{\{i_1,...,i_n\} = \{1,...,n\}}$
            $a_{1i_1}...a_{ni_n}(-1)^I$
    \end{proof}

    \newpage



    \begin{wtheorem}{det() function}{16cm}
        Function det: $M_{n \times n}(\mathbb{R})$ $\rightarrow$ $\mathbb{R}$
        is a determinant function

        \hspace{0.5cm}
        det(A) = $\sum_{\{i_1,...,i_n\} = \{1,...,n\}}$
                    $a_{1i_1}...a_{ni_n}(-1)^I$
        
        where I is the number of inversions
        in $(1,i_1),...,(n,i_n)$
    \end{wtheorem}

    \begin{proof}
        For j $\in$ \{1,...,n\},
        let x,y $\in$ $\mathbb{R}^n$ and $c_1,c_2$ $\in$ $\mathbb{R}$:

        \hspace{0.5cm}
        T($c_1x+c_2y$)
        = det($A_1,...,A_{j-1},c_1x+c_2y,A_{j+1},...,A_n$)

        \hspace{2.9cm}
        = $\sum_{\{i_1,...,i_n\} = \{1,...,n\}}$
            $a_{1i_1}...a_{(j-1)i_{j-1}}
            (c_1x+c_2y)_{ji_j}a_{(j+1)i_{j+1}}...a_{ni_n}(-1)^I$

        \hspace{2.9cm}
        = $c_1\sum_{\{i_1,...,i_n\} = \{1,...,n\}}$
            $a_{1i_1}...a_{(j-1)i_{j-1}}x_{ji_j}a_{(j+1)i_{j+1}}...a_{ni_n}(-1)^I$

        \hspace{3.4cm}
        + $c_2\sum_{\{i_1,...,i_n\} = \{1,...,n\}}$
            $a_{1i_1}...a_{(j-1)i_{j-1}}y_{ji_j}a_{(j+1)i_{j+1}}...a_{ni_n}(-1)^I$

        \hspace{2.9cm}
        = $c_1$det($A_1,...,A_{j-1},x,A_{j+1},...,A_n$)
            + $c_2$det($A_1,...,A_{j-1},y,A_{j+1},...,A_n$)

        \hspace{2.9cm}
        = $c_1T(x) + c_2T(y)$

        \vspace{0.3cm}

        For A =
        $\begin{bmatrix}
            A_1 & ... & A_j & ... & A_k & ... & A_n
        \end{bmatrix}$,
        let B =
        $\begin{bmatrix}
            A_1 & ... & A_k & ... & A_j & ... & A_n
        \end{bmatrix}$.

        By {\color{red} theorem 4.1.3}, the $I_B$, number of inversions in B, is
        equal to $S_B$, the number of swaps in B to turn
        $e_{i_1},...e_{i_k},...e_{i_j},...,e_{i_n}$
        into $e_1,...,e_n$. Thus:

        \hspace{0.5cm}
        det(B)
        = $\sum_{\{i_1,...,i_n\} = \{1,...,n\}}$
            $a_{1i_1}...a_{ki_k}...a_{ji_j}...a_{ni_n}(-1)^{I_B}$

        \hspace{1.85cm}
        = $\sum_{\{i_1,...,i_n\} = \{1,...,n\}}$
            $a_{1i_1}...a_{ji_j}...a_{ki_k}...a_{ni_n}(-1)^{S_B}$

        Note $e_{i_1},...e_{i_k},...e_{i_j},...,e_{i_n}$
        can first swap into $e_{i_1},...e_{i_j},...e_{i_k},...,e_{i_n}$
        and then perform swaps to turn into $e_1,...,e_n$.
        Since the number of swaps to turn
        $e_{i_1},...e_{i_j},...e_{i_k},...,e_{i_n}$ into $e_1,...,e_n$
        is $S_A$, the number of swaps in A, then $S_B$ = $S_A + 1$. Thus:

        \hspace{0.5cm}
        det(B)
        = $\sum_{\{i_1,...,i_n\} = \{1,...,n\}}$
            $a_{1i_1}...a_{ji_j}...a_{ki_k}...a_{ni_n}(-1)^{S_A+1}$

        \hspace{1.85cm}
        = $-\sum_{\{i_1,...,i_n\} = \{1,...,n\}}$
            $a_{1i_1}...a_{ji_j}...a_{ki_k}...a_{ni_n}(-1)^{S_A}$
        = -det(A)

        \vspace{0.3cm}

        For det($I_{n \times n}$),
        the only nonzero $a_{1i_1}...a_{ni_n}$
        is when each $a_{ki_k}$ = 1 else $a_{ki_k}$ = 0.
        Since there are no inversions from $e_1,...,e_n$ to $e_1,...,e_n$,
        then det($I_{n \times n}$) = $a_{1i_1}...a_{ni_n}(-1)^I$
        = $1...1(-1)^0$ = 1.
    \end{proof}

    \vspace{0.5cm}



    \begin{wtheorem}{det($A^T$) = det(A)}{16cm}
        For A $\in$ $M_{n \times n}(\mathbb{R})$, then
        det($A^T$) = det(A)
    \end{wtheorem}

    \begin{proof}
        Let det($A^T$) =
        $\sum_{\{i_1,...,i_n\} = \{1,...,n\}}$
            $a_{1i_1}...a_{ni_n}(-1)^{I_{A^T}}$.
        
        For $a_{k_1i_{k_1}}$,$a_{k_2i_{k_2}}$
        where $i_{k_1}$ $<$ $i_{k_2}$, suppose $k_1$ $<$ $k_2$.
        Then, there is no inversion for $a_{k_1i_{k_1}}$
        and $a_{k_2i_{k_2}}$.
        Then for $a_{i_{k_1}k_1}$,$a_{i_{k_2}k_2}$
        where $k_1$ $<$ $k_2$ and $i_{k_1}$ $<$ $i_{k_2}$,
        then there is no inversion.

        For $a_{k_1i_{k_1}}$,$a_{k_2i_{k_2}}$
        where $i_{k_1}$ $<$ $i_{k_2}$, suppose $k_1$ $>$ $k_2$.
        Then, there is an inversion for $a_{k_1i_{k_1}}$
        and $a_{k_2i_{k_2}}$.
        Then for $a_{i_{k_2}k_2}$,$a_{i_{k_1}k_1}$
        where $k_2$ $<$ $k_1$ and $i_{k_2}$ $>$ $i_{k_1}$,
        then there is an inversion.

        Thus, transpose perserves inversions
        so $I_{A^T}$ = $I_A$.

        Since each $a_{1i_1}...a_{ni_n}$ in $A^T$
        can be arrange in order by $i_1,...,i_n$
        into $b_{1i_1}...b_{ni_n}$
        which when transposed is $b_{i_11}...b_{i_nn}$
        which is in A. Thus:

        \hspace{0.05cm}
        det($A^T$) =
        $\sum_{\{i_1,...,i_n\} = \{1,...,n\}}$
            $a_{1i_1}...a_{ni_n}(-1)^{I_{A^T}}$
        = $\sum_{\{i_1,...,i_n\} = \{1,...,n\}}$
            $b_{i_11}...b_{i_nn}(-1)^{I_A}$
        = det(A)
    \end{proof}

    \newpage



    \begin{wtheorem}{Cofactor Expansion: Determing det(A) by parts}{16cm}
        For A $\in$ $M_{n \times n}(\mathbb{R})$, let
        let $A_{ij}$ be A, but the i-th row and j-th column removed.
        
        For any i-th row where i $\in$ \{1,...,n\}, then by fixing i:

        \hspace{0.5cm}
        det(A) = $\sum_{j=1}^n$ $(-1)^{i+j}a_{ij}$det($A_{ij}$)

        Or for any j-th column where j $\in$ \{1,...,n\}, then by fixing j:

        \hspace{0.5cm}
        det(A) = $\sum_{i=1}^n$ $(-1)^{i+j}a_{ij}$det($A_{ij}$)
    \end{wtheorem}

    \begin{proof}
        There are n possibles $a_{ij}$ choices in the first column and by choosing
        any such one, then that row is eliminated for choice in the following
        columns. Thus, there are n-1 possible $a_{ij}$ choices in the second column
        and by choosing any such one, then that row is also eliminated for choice
        in the following columns. Repeating the pattern, then
        there are n*(n-1)*(n-2)*...*1 = n! total unique
        $a_{1i_1},...,a_{ni_n}$ combinations.
        In the cofactor expansion, choose a fixed i. The case for a fixed j
        is analogous.
        For a fixed i, the cofactor expansion iterates through each of the n
        columns in row i so there are n unique $a_{ij}$.
        For each $a_{ij}$, the $A_{ij}$ has the i-th row and j-th column
        removed so $A_{ij}$ is a (n-1) by (n-1) matrix and thus, there
        are (n-1)! unique $a_{1i_1},...,a_{ni_n}$ combinations as proved earlier.
        Since each $A_{ij}$ removes a different j-th column, then each
        $a_{1i_1},...,a_{ni_n}$ from different columns are unique.
        Thus, the n unique $a_{ij}$ has (n-1)! unique $a_{1i_1},...,a_{ni_n}$
        so there are n*(n-1)! = n! unique $a_{1i_1},...,a_{ni_n}$.
        Thus, the $a_{1i_1},...,a_{ni_n}$ in the cofactor expansion must be
        equivalent to the $a_{1i_1},...,a_{ni_n}$ in the original determinant.

        For the fixed i, lets fixed j $\in$ \{1,...,n\}:

        \hspace{0.5cm}
        $
        \begin{bmatrix}
            a_{1,1} & a_{1,2} & a_{1,3} & \hdots
                    & a_{1,j-1} & {\color{lblue} a_{1,j}}
                    & a_{1,j+1} & \hdots & a_{1,n} \\
            \vdots & \vdots & \vdots & \ddots
                    & \vdots & \vdots & \vdots & \ddots & \vdots \\
            a_{i-1,1} & a_{i-1,2} & a_{i-1,3} & \hdots
                    & a_{i-1,j-1} & {\color{lblue} a_{i-1,j}}
                    & a_{i-1,j+1} & \hdots & a_{i-1,n} \\
            {\color{red} a_{i,1}} & {\color{red} a_{i,2}} & {\color{red} a_{i,3}}
                & \hdots & {\color{red} a_{i,j-1}} & {\color{purple} a_{i,j}}
                & {\color{red} a_{i,j+1}} & \hdots & {\color{red} a_{i,n}} \\
            a_{i+1,1} & a_{i+1,2} & a_{i+1,3} & \hdots
                    & a_{i+1,j-1} & {\color{lblue} a_{i+1,j}}
                    & a_{i+1,j+1} & \hdots & a_{i+1,n} \\
            \vdots & \vdots & \vdots & \ddots
                    & \vdots & \vdots & \vdots & \ddots & \vdots \\
            a_{n,1} & a_{n,2} & a_{n,3} & \hdots
                    & a_{n,j-1} & {\color{lblue} a_{n,j}}
                    & a_{n,j+1} & \hdots & a_{n,n} \\
        \end{bmatrix}
        $

        In the original determinant, each $a_{1i_1},...,a_{ni_n}$ associates
        $(-1)^{\text{\tiny \# inversions $a_{1i_1},...,a_{ni_n}$}}$.
        As proven earlier, each $a_{1i_1},...,a_{ni_n}$ is expressed
        in the coefactor expansion.
        So for any $a_{1i_1},...,a_{ni_n}$ that contains $a_{ij}$ with the
        fixed i,j, then from the $a_{ij}\text{det}(A_{ij})$ in the
        cofactor expansion, the det($A_{ij}$)
        consists of the other $a_{ij}$ in the $a_{1i_1},...,a_{ni_n}$
        since none of the other $a_{ij}$ can exist in row i or column j
        by definition of the determinant and thus, det($A_{ij}$) must account
        for all the inversions exclusively between the other $a_{ij}$.
        To account for the inversions between the other $a_{ij}$
        and the fixed $a_{ij}$, refer to the matrix above.
        The only $a_{ij}$ which contributes an inversion with the fixed $a_{ij}$
        must be in the lower left and upper right of the matrix
        by defintion of the determinant.
        Let A = \#$a_{ij}$ in upper left, B = \#$a_{ij}$ in upper right,
        C = \#$a_{ij}$ in lower left, and D = \#$a_{ij}$ in lower right.
        Since each $a_{1i_1},...,a_{ni_n}$
        must have a $a_{ij}$ in each row and column, then:

        \hspace{0.5cm}
        A+B = i - 1
        \hspace{1cm}
        A+C = j - 1
        \hspace{1cm}
        $\Rightarrow$
        \hspace{1cm}
        B+C = i + j - 2 - 2A

        \hspace{0.5cm}
        $(-1)^{I_A}$ = $(-1)^{B+C}$ = $(-1)^{i+j-2-2A}$
        = $(-1)^{i+j}(-1)^{-2}(-1)^{-2A}$
        = $(-1)^{i+j}$

        Thus:

        \hspace{0.5cm}
        det(A)
        = $\sum_{\{i_1,...,i_n\} = \{1,...,n\}}$
            $a_{1i_1}...a_{ni_n}(-1)^{I_A}$

        \hspace{1.8cm}
        = $\sum_{j=1}^n$ $a_{ij}$det($A_{ij}$)$(-1)^{I_A}$
        = $\sum_{j=1}^n$ $a_{ij}$det($A_{ij}$)$(-1)^{i+j}$
    \end{proof}

    \newpage



    \begin{example}
        Find det(A) where A =
        \scriptsize
        $\begin{bmatrix}
            1 & 0 & 1 & 2 \\
            9 & 1 & 3 & 0 \\
            9 & 2 & 2 & 0 \\
            5 & 0 & 0 & 3
        \end{bmatrix}$
        \normalsize
        using cofactor expansion
    \end{example}

    \begin{tbox}
        Using cofactor expansion along the 4th row:

        \hspace{0.5cm}
        det(A)
        = -5det($\begin{bmatrix}
                    0 & 1 & 2 \\
                    1 & 3 & 0 \\
                    2 & 2 & 0    
                \end{bmatrix}$)
        + 0det($\begin{bmatrix}
                    1 & 1 & 2 \\
                    9 & 3 & 0 \\
                    9 & 2 & 0    
                \end{bmatrix}$)
        - 0det($\begin{bmatrix}
                    1 & 0 & 2 \\
                    9 & 1 & 0 \\
                    9 & 2 & 0    
                \end{bmatrix}$)
        + 3det($\begin{bmatrix}
                    1 & 0 & 1 \\
                    9 & 1 & 3 \\
                    9 & 2 & 2    
                \end{bmatrix}$)

        \hspace{1.9cm}
        = -5(4-12) + 3(2+18-9-6)
        = 40 + 15 = 55
    \end{tbox}

    \vspace{0.5cm}





\subsection{ Properties of the Determinant }

    \begin{wtheorem}{Relationship between Determinant and Elementary
    row operations}{16cm}
        Let A,B $\in$ $M_{n \times n}(\mathbb{R})$.
    \end{wtheorem}

    \begin{enumerate}[label=(\alph*), leftmargin=2cm, itemsep=0.1cm]
        \item {\color{lgreen} Row or Column Multiplication}

            Let B be obtained by multiplying the i-th row or j-th column
            of A by c $\in$ $\mathbb{R}$. Then:

            \hspace{0.5cm}
            det(B) = c det(A)

            \begin{proof}[15cm]
                Let B be A with the i-th column multiplied by c. Then:
                
                \hspace{0.5cm}
                det(B)
                = det($B_1,...,B_i,...B_n$)
                = det($A_1,...,cA_i,...A_n$)

                \hspace{1.8cm}
                = c det($A_1,...,A_i,...A_n$)
                = c det(A)

                Let B be A with the j-th column multiplied by c. Then:

                \hspace{0.5cm}
                det(B)
                = det($B^T$)
                = c det($A^T$)
                = c det(A)
            \end{proof}

        \item {\color{lgreen} Row or Column Addition}
        
            Let B be obtained by adding to the i-th row by c $\in$ $\mathbb{R}$
            times the j-th row of A or by adding to the i-th column by
            c $\in$ $\mathbb{R}$ times the j-th column of A. Then:

            \hspace{0.5cm}
            det(B) = det(A)
            
            \begin{proof}[15cm]
                Let B be A with the i-th column added by c times the j-th column.
                Then:
                
                \hspace{0.5cm}
                det(B)
                = det($B_1,...,B_i,...,B_j,...,B_n$)
                = det($A_1,...,A_i+cA_j,...,A_j,...,A_n$)

                \hspace{1.8cm}
                = det($A_1,...,A_i,...,A_j,...,A_n$)
                    + c det($A_1,...,A_j,...,A_j,...,A_n$)

                \hspace{1.8cm}
                = det(A) + c0
                = det(A)

                Let B be A with the i-th row added by c times the j-th row.
                Then:

                \hspace{0.5cm}
                det(B)
                = det($B^T$)
                = det($A^T$)
                = det(A)
            \end{proof}
        
        \item {\color{lgreen} Row or Column Swapping}
        
            Let B be obtained swapping the i-th and j-th rows of A or
            by swapping the i-th and j-th columns of A. Then:

            \hspace{0.5cm}
            det(B) = -det(A)

            \begin{proof}[15cm]
                Let B be A with the i-th and j-th columns swapped. Then:

                \hspace{0.5cm}
                det(B)
                = det($B_1,...,B_i,...,B_j,...,B_n$)
                = det($A_1,...,A_j,...,A_i,...,A_n$)

                \hspace{1.8cm}
                = -det($A_1,...,A_i,...,A_j,...,A_n$)
                = -det(A)

                Let B be A with the i-th and j-th rows swapped. Then:

                \hspace{0.5cm}
                det(B)
                = det($B^T$)
                = -det($A^T$)
                = -det(A)
            \end{proof}
    \end{enumerate}

    \newpage



    \begin{example}
        Find det(A) where A =
        \scriptsize
        $\begin{bmatrix}
            1 & 0 & 1 & 2 \\
            9 & 1 & 3 & 0 \\
            9 & 2 & 2 & 0 \\
            5 & 0 & 0 & 3
        \end{bmatrix}$
        \normalsize
        using elementary row operations
    \end{example}

    \begin{tbox}
        \scriptsize
        $\begin{bmatrix}
            1 & 0 & 1 & 2 \\
            9 & 1 & 3 & 0 \\
            9 & 2 & 2 & 0 \\
            5 & 0 & 0 & 3
        \end{bmatrix}$
        $\Rightarrow$
        $\begin{bmatrix}
            1 & 0 & 1 & 2 \\
            0 & 1 & -6 & -18 \\
            0 & 2 & -7 & -18 \\
            0 & 0 & -5 & -7
        \end{bmatrix}$
        $\Rightarrow$
        $\begin{bmatrix}
            1 & 0 & 1 & 2 \\
            0 & 1 & -6 & -18 \\
            0 & 0 & 5 & 18 \\
            0 & 0 & -5 & -7
        \end{bmatrix}$
        $\Rightarrow$
        $\begin{bmatrix}
            1 & 0 & 1 & 2 \\
            0 & 1 & -6 & -18 \\
            0 & 0 & 5 & 18 \\
            0 & 0 & 0 & 11
        \end{bmatrix}$
        \normalsize
        = B

        Since only row additions were performed,
        then det(A) = det(B) = 1(1)(5)(11) = 55.
    \end{tbox}

    \vspace{0.5cm}



    \begin{wtheorem}{Invertible A $\Leftrightarrow$ det(A) $\not =$ 0}{16cm}
        Let A $\in$ $M_{n \times n}(\mathbb{R})$.
        Then, A is invertible if and only if det(A) $\not =$ 0.
    \end{wtheorem}

    \begin{proof}
        By {\color{red} theorem 2.4.5}, A is invertible if and only if
        rref(A) = $I_{n \times n}$.
        Then, there is a sequence of elementary row operations that
        transformation A into $I_{n \times n}$.
        By {\color{red} theorem 4.2.1}, then:

        \hspace{0.5cm}
        det(A) = $(-1)^S$det($I_{n \times n}$) = $(-1)^S$ $\not =$ 0

        where S is the number of row swaps in the sequence.
        Then if det(A) = 0, then A cannot be invertible.
        Thus, A is invertible if and only if det(A) $\not =$ 0.
    \end{proof}

    \vspace{0.5cm}



    \begin{wtheorem}{det(AB) = det(A)det(B)}{16cm}
        For A,B $\in$ $M_{n \times n}(\mathbb{R})$, then
        det(AB) = det(A)det(B)
    \end{wtheorem}

    \begin{proof}
        Suppose at least one of A,B is not invertible.
        Let A be not invertible.

        Then by {\color{red} theorem 4.2.2}, det(A) = 0.
        Since A is not invertible, then AB is not invertible.

        \hspace{0.5cm}
        det(AB) = 0 = 0det(B) = det(A)det(B)

        Suppose A,B are invertible. By {\color{red} theorem 2.4.5},
        rref(A) = rref(B) = $I_{n \times n}$.
        Thus, there is a sequence $S_1$ of elementary row operations
        that transforms A into $I_{n \times n}$.
        Thus, sequence $S_1$ will transform AB into $I_{n \times n}$B = B.
        And then there is another sequence $S_2$
        of elementary row operations that transforms B into $I_{n \times}$.
    
        \hspace{0.5cm}
        det(A) = $(-1)^{S_A}$
        \hspace{1cm}
        det(A) = $(-1)^{S_B}$

        where $S_A$ are the number of row swaps in $S_1$
        and $S_B$ are the number of row swaps in $S_2$.

        \hspace{0.5cm}
        det(AB)
        = $(-1)^{S_A}$det(B)
        = $(-1)^{S_A} (-1)^{S_B}$
        = det(A)det(B)
    \end{proof}

    \vspace{0.5cm}



    \begin{corollary}{det($A^k$) = (det(A))$^k$}{16cm}
        For A $\in$ $M_{n \times n}(\mathbb{R})$,
        then det($A^k$) = (det(A))$^k$
    \end{corollary}

    \begin{proof}
        det($A^k$)
        = det(A)det($A^{k-1}$)
        = det(A)det(A)det($A^{k-2}$)
        = ...
        = (det(A))$^k$
    \end{proof}

    \vspace{0.5cm}



    \begin{corollary}{det($A^{-1}$) = (det(A))$^{-1}$}{16cm}
        For invertible A $\in$ $M_{n \times n}(\mathbb{R})$,
        then det($A^{-1}$) = (det(A))$^{-1}$
    \end{corollary}

    \begin{proof}
        Since A is invertible, then $A^{-1}A$ = $I_{n \times n}$. Then:

        \hspace{0.5cm}
        1 = det($I_{n \times n}$)
        = det($A^{-1}A$)
        = det($A^{-1}$)det(A)
        \hspace{0.5cm}
        $\Rightarrow$
        \hspace{0.5cm}
        det($A^{-1}$) = (det(A))$^{-1}$.
    \end{proof}

    \newpage



    \begin{corollary}{Similar matrices have the same Determinant}{16cm}
        For similar A,B $\in$ $M_{n \times n}(\mathbb{R})$,
        then det(A) = det(B)
    \end{corollary}

    \begin{proof}
        Since A,B are similar, then there is a invertible
        X $\in$ $M_{n \times n}(\mathbb{R})$ such that AX = XB. Then:

        \hspace{0.5cm}
        det(A)det(X)
        = det(AX)
        = det(XB)
        = det(X)det(B)

        Since X is invertible, then det(X) $\not =$ 0 so det(A) = det(B).
    \end{proof}

    \vspace{0.5cm}



    \begin{corollary}{Determinant of Orthogonal matrices}{16cm}
        For orthogonal A $\in$ $M_{n \times n}(\mathbb{R})$,
        then det(A) = $\pm 1$.
    \end{corollary}

    \begin{proof}
        Since A is orthogonal, then by {\color{red} theorem 3.3.2},
        $A^TA$ = $I_{n \times n}$. Then:

        \hspace{0.5cm}
        1 = det($I_{n \times n}$)
        = det($A^TA$)
        = det($A^T$)det(A)
        = (det(A))$^2$
        \hspace{0.5cm}
        $\Rightarrow$
        \hspace{0.5cm}
        det(A) = $\pm 1$
    \end{proof}

    \vspace{0.5cm}





\subsection{ Volume }

    \begin{definition}{Volume of Parallelotope}{16cm}
        Let $v_1,...,v_n$ $\in$ $\mathbb{R}^n$.
        Then a {\color{lblue} parallelotope}, P($v_1,...,v_n$),
        is a n-th order parallelogram.

        For i = \{2,...,n\}, let
        $v_i^{\perp}$ = $v_i$ - $\text{proj}_{\text{span}(v_1,...,v_{i-1})}v_i$.
        The volume of a parallelotope:

        \hspace{0.5cm}
        $\text{Vol}_n(P(v_1,...,v_n))$
        = $|v_1| \ |v_2^{\perp}| \ |v_3^{\perp}| \ ... \ |v_n^{\perp}|$
    \end{definition}

    \vspace{0.5cm}



    \begin{wtheorem}{$|$det($v_1,...,v_n$)$|$ = $\text{Vol}_n(P(v_1,...,v_n))$}{16cm}
        For $v_1,...,v_n$ $\in$ $\mathbb{R}^n$,
        let A =
        $\begin{bmatrix}
            v_1 & ... & v_n 
        \end{bmatrix}$
        $\in$ $M_{n \times n}(\mathbb{R})$. Then:

        \hspace{0.5cm}
        $\text{Vol}_n(P(v_1,...,v_n))$
        = $|\text{det}(A)|$
    \end{wtheorem}

    \begin{proof}
        By {\color{red} theorem 3.2.9},
        $v_1,v_2^{\perp},...,v_n^{\perp}$ $\in$ $\mathbb{R}^n$
        is orthogonal.
        Let B =
        $\begin{bmatrix}
            v_1 & v_2^{\perp} & ... & v_n^{\perp}
        \end{bmatrix}$. Thus:

        \hspace{0.5cm}
        $B^TB$ =
        \scriptsize
        $\begin{bmatrix}
            v_1^T \\
            (v_2^{\perp})^T \\
            \vdots \\
            (v_n^{\perp})^T
        \end{bmatrix}
        \begin{bmatrix}
            v_1 & v_2^{\perp} & \hdots & v_n^{\perp}
        \end{bmatrix}$ =
        $\begin{bmatrix}
            v_1^Tv_1 & v_1^Tv_2^{\perp} & \hdots & v_1^Tv_n^{\perp} \\
            (v_2^{\perp})^Tv_1 & (v_2^{\perp})^Tv_2^{\perp}
                & \hdots & (v_2^{\perp})^Tv_n^{\perp} \\
            \vdots & \vdots & \ddots & \vdots \\
            (v_n^{\perp})^Tv_1 & (v_n^{\perp})^Tv_2^{\perp}
                & \hdots & (v_n^{\perp})^Tv_n^{\perp}
        \end{bmatrix}$ =
        $\begin{bmatrix}
            |v_1|^2 & 0 & \hdots & 0 \\
            0 & |v_2^{\perp}|^2 & \hdots & 0 \\
            \vdots & \vdots & \ddots & \vdots \\
            0 & 0 & \hdots & |v_n^{\perp}|^2
        \end{bmatrix}$
        \normalsize

        Since each $v_i^{\perp}$ = $v_i - (c_{i1}v_1 + .... + c_{i(i-1)}v_{i-1})$
        for $c_{i1},...,c_{i(i-1)}$ $\in$ $\mathbb{R}$, then
        by {\color{red} theorem 4.2.1}:

        \hspace{0.3cm}
        det(B)
        = det($v_1,v_2^{\perp},v_3^{\perp},...,v_n^{\perp}$)
        = det($v_1,v_2,v_3^{\perp},...,v_n^{\perp}$)
        = ...
        = det($v_1,v_2,v_3,...,v_n$)
        = det(A)

        Thus:

        \hspace{0.5cm}
        $|\text{det}(A)|$
        = $\sqrt{\text{det}(A)\text{det}(A)}$
        = $\sqrt{\text{det}(A^T)\text{det}(A)}$
        = $\sqrt{\text{det}(A^TA)}$
        = $\sqrt{\text{det}(B^TB)}$

        \hspace{2.1cm}
        = $\sqrt{|v_1|^2 |v_2^{\perp}|^2 ... |v_n^{\perp}|^2}$
        = $|v_1| |v_2^{\perp}| ... |v_n^{\perp}|$
        = $\text{Vol}_n(P(v_1,...,v_n))$
    \end{proof}

    \vspace{0.5cm}



    \begin{corollary}{Expansion Factor:
    Linear transformation of a Parallelotope}{16cm}
        Let linear transformation T: $\mathbb{R}^n$ $\rightarrow$ $\mathbb{R}^n$
        be T(x) = Ax. Then for $v_1,...,v_n$ $\in$ $\mathbb{R}^n$,
        T(P($v_1,...,v_n$)) is a parallelotope where:

        \hspace{0.5cm}
        $\text{Vol}_n(T(P(v_1,...,v_n)))$
        = $|\text{det}(A)| \text{Vol}_n(P(v_1,...,v_n))$

        So, for T(x) = Ax, $|\text{det}(A)|$ is called the
        {\color{lblue} expansion factor} of T.
    \end{corollary}

    \begin{proof}
        For any v $\in$ $P(v_1,...,v_n)$, then
        v = $c_1v_1 + ... + c_nv_n$ for $c_1,...,c_n$ $\in$ [0,1].
        Thus:

        \hspace{0.5cm}
        $T(P(v_1,...,v_n))$
        = \{ $A(c_1v_1 + ... + c_nv_n)$ \}
        = \{ $c_1Av_1 + ... + c_nAv_n$ \}

        Thus, T(P($v_1,...,v_n$)) is the parallelotope,
        P($Av_1,...,Av_n$).
        By {\color{red} theorem 4.3.2}:

        \hspace{0.3cm}
        \small
        $\text{Vol}_n(T(P(v_1,...,v_n)))$
        = $|\text{det}([Av_1 \ ... \ Av_n])|$
        = $|\text{det}(A)\text{det}([v_1 \ ... \ v_n])|$
        = $|\text{det}(A)| \text{Vol}_n(P(v_1,...,v_n))$
    \end{proof}

    \newpage



    \begin{wtheorem}{Cauchy-Binet Formula}{16cm}
        For k $\leq$ n, let $v_1,...,v_k$ $\in$ $\mathbb{R}^n$.
        Let A =
        $\begin{bmatrix}
            v_1 & ... & v_k
        \end{bmatrix}$
        $\in$ $M_{n \times k}(\mathbb{R})$. Then:

        \hspace{0.5cm}
        $\text{Vol}_k(P(v_1,...,v_k))$
        = $\sqrt{\text{det}(A^TA)}$
    \end{wtheorem}

    \begin{proof}
        Since each $v_i^{\perp}$ = $v_i - (c_{i1}v_1 + .... + c_{i(i-1)}v_{i-1})$
        for $c_{i1},...,c_{i(i-1)}$ $\in$ $\mathbb{R}$, then:

        \footnotesize
        \hspace{0.5cm}
        B =
        $\begin{bmatrix}
            v_1 & v_2^{\perp} & ... & v_k^{\perp}
        \end{bmatrix}$ =
        $\begin{bmatrix}
            v_1 & v_2 & ... & v_k
        \end{bmatrix}
        \begin{bmatrix}
            1 & -c_{21} & -c_{31} & \hdots & -c_{k1} \\
            0 & 1 & -c_{32} & \hdots & -c_{k2} \\
            0 & 0 & 1 & \hdots & -c_{k3} \\
            \vdots & \vdots & \vdots & \ddots & \vdots \\
            0 & 0 & 0 & \hdots & 1
        \end{bmatrix}$
        = AX
        \normalsize

        Since det(X) = 1, then by {\color{red} theorem 4.2.2}, X is invertible
        so $BX^{-1}$ = $AXX^{-1}$ = A.
        By {\color{red} theorem 3.2.9},
        $v_1,v_2^{\perp},...,v_n^{\perp}$ $\in$ $\mathbb{R}^n$
        is orthogonal. Thus:

        \footnotesize
        $B^TB$ =
        $\begin{bmatrix}
            v_1^T \\
            (v_2^{\perp})^T \\
            \vdots \\
            (v_k^{\perp})^T
        \end{bmatrix}
        \begin{bmatrix}
            v_1 & v_2^{\perp} & \hdots & v_k^{\perp}
        \end{bmatrix}$ =
        $\begin{bmatrix}
            v_1^Tv_1 & v_1^Tv_2^{\perp} & \hdots & v_1^Tv_k^{\perp} \\
            (v_2^{\perp})^Tv_1 & (v_2^{\perp})^Tv_2^{\perp}
                & \hdots & (v_2^{\perp})^Tv_k^{\perp} \\
            \vdots & \vdots & \ddots & \vdots \\
            (v_k^{\perp})^Tv_1 & (v_k^{\perp})^Tv_2^{\perp}
                & \hdots & (v_k^{\perp})^Tv_k^{\perp}
        \end{bmatrix}$ =
        $\begin{bmatrix}
            |v_1|^2 & 0 & \hdots & 0 \\
            0 & |v_2^{\perp}|^2 & \hdots & 0 \\
            \vdots & \vdots & \ddots & \vdots \\
            0 & 0 & \hdots & |v_k^{\perp}|^2
        \end{bmatrix}$
        \normalsize

        \hspace{0.5cm}
        $\sqrt{\text{det}(A^TA)}$
        = $\sqrt{\text{det}((BX^{-1})^T(BX^{-1}))}$
        = $\sqrt{\text{det}((X^{-1})^T) \text{det}(B^TB) \text{det}(X^{-1})}$

        \hspace{2.8cm}
        = $\sqrt{\text{det}((X^{-1})) \text{det}(B^TB) \text{det}(X)^{-1}}$
        = $\sqrt{\text{det}(B^TB)}$
        
        \hspace{2.8cm}
        = $\sqrt{|v_1|^2 |v_2^{\perp}|^2 ... |v_k^{\perp}|^2}$
        = $|v_1| |v_2^{\perp}| ... |v_k^{\perp}|$
        = $\text{Vol}_k(P(v_1,...,v_k))$
    \end{proof}

    \vspace{0.5cm}



    \begin{example}
        Find the volume of the parallelotope with
        sides (1,0,0,0), (1,1,1,1), and (1,2,3,4)
    \end{example}

    \begin{tbox}
        Let A =
        $\begin{bmatrix}
            1 & 1 & 1 \\
            0 & 1 & 2 \\
            0 & 1 & 3 \\
            0 & 1 & 4
        \end{bmatrix}$.
        Then,
        $A^TA$ =
        $\begin{bmatrix}
            1 & 0 & 0 & 0 \\ 
            1 & 1 & 1 & 1 \\
            1 & 2 & 3 & 4
        \end{bmatrix}
        \begin{bmatrix}
            1 & 1 & 1 \\
            0 & 1 & 2 \\
            0 & 1 & 3 \\
            0 & 1 & 4
        \end{bmatrix}$ =
        $\begin{bmatrix}
            1 & 1 & 1 \\
            1 & 4 & 10 \\
            1 & 10 & 30
        \end{bmatrix}$.

        \hspace{0.5cm}
        $\begin{bmatrix}
            1 & 1 & 1 \\
            1 & 4 & 10 \\
            1 & 10 & 30
        \end{bmatrix}$
        $\Rightarrow$
        $\begin{bmatrix}
            1 & 1 & 1 \\
            0 & 3 & 9 \\
            0 & 9 & 29
        \end{bmatrix}$
        $\Rightarrow$
        $\begin{bmatrix}
            1 & 1 & 1 \\
            0 & 3 & 9 \\
            0 & 0 & 2
        \end{bmatrix}$

        The volume is
        $\sqrt{\text{det}(A^TA)}$
        = $\sqrt{1(3)(2)}$
        = $\sqrt{6}$.
    \end{tbox}




